\chapter{Metodologia}
Després d'haver fet la recerca previa, i d'haver decidit fer una xarxa neuronal ``model'' que compleixi tots els requesits que esperavem, ara cal pensar amb quina metodologia podrem aconseguir el nostre objectiu. Per tant vol dir que hem de fer un altre recerca en el que utilizarem les millors eines per fer el Treball de Recerca de manera ràpida i eficient.

Primer ens hem fet la pregunta de sobre quins àmbits hem de recorre, una vegada feta la pregunta ja buscarem les millors eines o mètodes que ens ajudarà en cadascún dels àmbits. A continuació farem unes presentacions dels àmbits que vam tractar i les diferents solucions que vam donar i el definitiu.

Per començar tractem  l'apartat \ref{4.1}, que consisteix en com vam estructurar la via de comunicació entre nosaltres i el tutor. A l'apartat \ref{4.2}, tracta de quin editor de text vam escollir, que es un punt crucial per el treball ja que moltes vegades les eines que ens aporta l'editor en si mateix ens pot estalviar molt de temps i fer que el treball sigui visualment més accessible.




\section{Comunicació}\label{4.1}
\section{Editor de text}\label{4.2}

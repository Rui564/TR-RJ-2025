\chapter{Metodologia}
Després d'haver fet la recerca previa, i d'haver decidit fer una xarxa neuronal ``model'' que compleixi tots els requesits que esperavem, ara cal pensar amb quina metodologia podrem aconseguir el nostre objectiu. Per tant vol dir que hem de fer un altre recerca en el que utilizarem les millors eines per fer el Treball de Recerca de manera ràpida i eficient.


Primer ens em fet la pregunta de  sobre quins àmbits hem de recorre, una vegada feta la pregunta ja buscarem les millors eines o mètode que ens ajudarà en cadascún dels àmbits. A continuació farem unes presentacions dels àmbits que vam tractar i les diferents solucions que vam donar i el definitiu.

Per començar tractem  l'apartat \ref{sec:4.1}, que consisteix en com vam estructurar la via de comunicació entre nosaltres i el tutor. A l'apartat \ref{sec:4.2}, tracta de quin editor de text vam escollir, que es un punt crucial per el treball ja que les eines que ens aporta l'editor en si mateix ens pot estalviar molt de temps i convertirlo visualment més accessible. A continuació hi es l'apartat \ref{sec:4.3}, on t'explica la plataforma que hem escollit per l'entorn col·laboratiu. Finalment per poder crear una xarxa neuronal hem d'utilizar un llenguatge de programació i això s'explicarà a l'apartat \ref{sec:4.4}.



\section{Comunicació}\label{sec:4.1}
Com que som dos persones fent el treball amb un tutor, ens és fonamental tenir una bona via comunicativa. Per tant, hem acollit diferents maneres de comunicació. Ara bé no em pogut registrar tots els registres que hem tingut però la gran majoria s'ha pogut conservar.\\
D'aquí cap avall explicarem les diferents maneres de comunicació utilizades en l'elaboracio d'aquest treball:\\
\subsection{Full de Calculs}
Hem realitzat diverses reunions amb el tutor, i la via que vam utilitzar per configurar l'horari de les quedades és a travès d'un full de càlcul. En aquestes quedades ens em organitzat el treball i ens em quedat amb els diferents mètodes que utilizariem per el trebal, el tutor ens arregla qualsevol tipus de problema, i la instalació dels programes necessaris.
\subsection{Correu Electronica}
L'eina que vam utilizar a distància és el servei de correu electronic Gmail. Vam fer ús d'aquest aparell per el simple fet de que és simple i eficient,i té una gran memòria d'emmagatzament per conservar totes les conserves que em fet.
Les converses fetes per el correu electrònic ha sigut la principal via de comunicació que em establert, per la facilitat i l'eficiència que ens aporta. Per tant, el tema ensenyament i dubtes simples o feiem tot per gmail.

\subsection{Git}
També amb l'ajuda del git, un software que ens va ajudar a guardar tots els canvis que hem anat fent durant el treball de recerca. Vam  establir una comunicació diària amb el progrés del treball, deixant comentaris dels canvis i adaptacions que vam anar fet.


\section{Editor de text i processador de text}\label{sec:4.2}
Per aconseguir una bona presentació de TR, haviem d'escollir o un editor de text o un processador. Tant l'editor com el processador són per crear documents, pero cadascún te les seves funcions.\\
L'editor del text, és tal com el seu nom, serveix per editar els documents. És molt simple i això genera inconvenients, com per exemple no poder utilizar formats avançat o una personificació. Però, les avantatge d'aquest és que pots tenir la capacitat de navegar rapidament de fitxer en fitxer, facilita la codificació i té compabilitat amb tots els tipus de llenguatge. És l'ideal per programadors.\\
Un processador de text són tipus de programes més elaborats i més complexos. Com que són programes amb tanta complexibilitat, et dona accès a formats avançat, utilitzat per a documents professionals com curriculums, llibres i d'altres més, també et donen tot tipus d'opcions col·laboratives com disseny de pàgina, correció gràfica i molt més.\\
Per a la part teòrica del nostre projecte sobre la Intel·ligència Artificial, vam seleccionar el format PDF degut a:
\begin{enumerate}
 \item Connexió segura i estable: Permet l'accés sense la dependència d'una connexió a Internet constant.
 \item Baixos requisits de recursos: Pot obrir-se en gairebé qualsevol dispositiu sense necessitat de programes específiques.
 \item Portabilitat: Manté el format independentment del sistema operatiu utilitzat.
 \item Multifuncionalitat: Admet text, fórmules matemàtiques, imatges i codi de programació integrat.
\end{enumerate}
L'estalvi de temps va ser un factor decisiu, ja que evita problemes de compatibilitat entre fórmules matemàtiques, imatges i el format del document.
En la nostra recerca d'eines que satisfacin tots els requisits, vam identificar dos tipus principals de processadors: Una que es deia WYSIWYG (What You See Is What You Get) i l'altre el WYSIWYM (What You See Is What You Mean). Ara donarem una explicació de les advantatges i desavantatges que tenen:
\subsection{What You See Is What You Get(WYSIWYG)}
WYSIWYG, es refereix a un tipus de processador que permet als usuaris veure en temps real el resultat final del document o disseny mentre editen, sense la necessitat de coneixer el codi o llenguatges de marcatges. Aquests tipus d'editors poden ser tant com el microsoft word o Google drive.

\textbf{Característiques}:
\begin{itemize}
 \item \textbf{Interfície visual:} Els canvis de format (negretes, taules, imatges) es fan mitjançant eines gràfiques, no codi.
 \item \textbf{Generació automàtica de codi:} El programa crea el codi subjacent (HTML, CSS, etc.) sense que l'usuari hi intervingui directament.
 \item \textbf{Ús generalitzat:} S'utilitza en àmbits com l'edició web, el disseny gràfic i la publicació digital.
\end{itemize}
Avantatges:
\begin{itemize}
\item \textbf{Edició visual intuïtiva:} Permet ajustar el format directament (arrossegar imatges, canviar fonts amb clics).
 \item Ideal per a usuaris sense coneixements tècnics.
\item \textbf{Resultat immediat:} No cal compilar, els canvis es veuen al moment.
\item \textbf{Eines integrades:} Funcions com correcció ortogràfica, taules gràfiques, o opcions de disseny accessibles des del menú
\item \textbf{Millor per a maquetació complexa:} Documents amb molts elements gràfics (pòsters, fulls informatius)
\end{itemize}
Desavantatges:
\begin{itemize}
 \item \textbf{Poca precisió en contingut tècnic:} Fórmules matemàtiques o codis es poden deformar-se a l'hora de canviar el format.
 \item \textbf{Errors en documents llargs:} El format manual pot provocar errors (salt de pàgina, numeració desorganitzada).
 \item \textbf{Dependència del programa:} Si s'obre en un altre software, el disseny pot variar.
 \item \textbf{Control de versions complicat:} Difícil treballar en equip sense conflictes de format.
\end{itemize}

\subsection{What You See Is What You Mean(WYSIWYM)}\label{subsec:4.2.2}
WSYSIWYM, es refereix a un tipus de processador que es centra en l'estructura del contigut, no en la seva aparença visual immediata, l'usuari marca el text segons la seva funció (títol, secció, cita i d'altres més) i el format final s'aplica mitjançant un full d'estil com css o Latex.
Avantatges:
\begin{itemize}
\item \textbf{Precisió en elements tècnics:} Fórmules matemàtiques, codi o referències es gestionen amb sintaxi clara.
\item \textbf{Consistència automàtica:} L'estil s'aplica globalment.
\item \textbf{Lleuger i portable:} Els fitxers són de text pla.
 \item \textbf{Ideal per a treball en entorns col·laboratius:} Es pot actualizar amb git i fusionar sense haber de canviar el format.
\end{itemize}
Desavantatges:
\begin{itemize}
\item \textbf{Corba d'aprenentatge:} Cal aprendre una sintaxi específica.
 \item \textbf{Previsualització no immediata:} En LaTeX, cal compilar; en Markdown, cal un renderitzador.
 \item \textbf{Limitacions en disseny gràfic:} Personalització avançada (ex: posicionament exacte d'imatges) requereix codi addicional.
\end{itemize}

\begin{table}[h!]
\begin{tabular}{|l|l|l|}
\hline
\textbf{Aspecte} & \textbf{WYSIWYG} & \textbf{WYSIWYM} \\ \hline
\textbf{Enfocament} & Aparença visual immediata & Estructura semàntica del contingut \\ \hline
\textbf{Exemples} & Microsoft Word, WordPress & LaTeX, LyX \\ \hline
\textbf{Usuaris} & No tècnics, dissenyadors &  Acadèmics, desenvolupadors tècnics \\ \hline
\textbf{Control} &  Limitada (codi generat automàtic) & Alt (definició manual de l'estructura) \\ \hline
\end{tabular}
\caption{Comparativa entre WYSIWYG i WYSIWYM}
\end{table}



font(WYSIWYG): (\href{https://es.wikipedia.org/wiki/WYSIWYG}{Wikipedia} \href{https://es.wikipedia.org/wiki/WYSIWYG}{Arimetrics} \href{https://www.wix.com/encyclopedia/definition/wysiwyg}{Wix}).
font(WYSIWYM): (\href{https://en.wikipedia.org/wiki/WYSIWYM}{Wikipedia} \href{https://en.ryte.com/wiki/WYSIWYG/}{Ryte WIKI}).



\subsection{\LaTeX}
\begin{comment}
% l'entorn figure et numera la figura i et permet posar una nota al peu de la figura, però costa controlar la posició
\begin{figure}[h!]
    \centering
    \includegraphics[width=0.5\textwidth]{./figures/latex.png}
    \caption{Logo del Latex}
\end{figure}%

% si ho fas d'aqesta manera
\begin{center}
    \includegraphics[width=0.75\textwidth]{./figures/latex.pdf}
\end{center}
\end{comment}

\LaTeX, és un tipus de WYSIWYG, on s'enfoca en l'estructura lògica, no la visual, tal com vam mencionar en l'apartat \ref{subsec:4.2.2}. LaTeX és un sistema de composició tipogràfica d'alta qualitat, especialment dissenyat per a la creació de documents acadèmics, tècnics i científics. A diferència dels processadors de text tradicionals com Microsoft Word, LaTeX es basa en codi per estructurar el document, separant el contingut de la seva presentació visual. Va ser desenvolupat per Leslie Lamport en 1984 com un conjunt de macros per al sistema TeX de Donald Knuth. La raó principal de la seva creació va ser la impotència dels processadors de text de l'època per compilar fórmules matemàtiques avançades i gestionar documents complexos amb precisió tipogràfica. És per aquesta raó la qual el vam escollir com a processador per satisfer els requeriments que demanem. Per més informació o dubte podeu consultar el manual de latex:  \href{https://manualdelatex.com/}{Manual de LaTeX}
\subsection{Kile}
Kile es un editor especific del \LaTeX que funciona en els sistemes operatius tant Linux, Windows com Apple Macintosh. L'editor va ser desenvolupat per la comunitat  \nameref{subsec:KDE} en que ofereix diversos eines avannçades per a l'edicio de LaTeX. Tot i que no es l'unic editor que existeix pero es el que més ens convé per el fet de que sigui creada per documents academics i cientifics.\\

Avantatges:
\begin{itemize}
\item \textbf{Entorn integrat complet:} Kile ofereix tot el necessari per treballar amb LaTeX en una sola interfície, incloent editor, compilador, visualitzador i gestor de referències
\item \textbf{Altament configurable: } Permet afegir botons a la barra d'eines per executar scripts personalitzats i automatitzar tasques complexes
\item \textbf{Feedback WYSWYG:} Et dona una sensació quasi WYSWYG, gracies als visualitzadors que els té integrades en que pots veure els canvis en temps reals.
\item \textbf{Eines avançades d'edició:} Té autocompletador de comandes, accés rapid a simbols matematics, navegacions rapides per seccions.
\end{itemize}
Desavantatges:
\begin{itemize}
\item \textbf{Nivell d'apranentatge:} Costa molt d'aprendre l'interficie si no estas familialitzat amb KDE
\item \textbf{Configuració Inicial:} Requereix molts tipus de configuracions externs per tal d'arrenca.
\end{itemize}

Ara mostrarem una taula de comparació respecte els altres editors:
LyX és més adequat per a principiants gràcies a la seva interfície WYSIWYG. No obstant, Kile es més flexible per als usuaris avançats i te més control en els codis i funcions. \\
%Taula 1
\begin{table}[h!]
 \begin{tabular}{|l|l|l|}
\hline
 \LaTeX & \textbf{Kile} & \textbf{LyX} \\ \hline
 \textbf{Facilitat d'ús} &  & X \\ \hline
 \textbf{Control i Flexibilitat} & X & \\ \hline
 \textbf{Visualització en temps real} & & X \\ \hline
\end{tabular}
\end{table}

TeXmaker i TeXstudio comparteixen moltes similituds amb Kile en termes de funcionalitat i d'interfície. \\

%Taula 2
\begin{table}[h!]
 \begin{tabular}{|l|l|l|}
\hline
 \LaTeX & \textbf{Kile} & \textbf{TeXmaker i TeXstudio} \\ \hline
 \textbf{similitud de funcionalitat} & - & - \\ \hline
 \textbf{Integració KDE} & X & \\ \hline
 \textbf{Disponibilitat dels sistemes operatius} & X &  \\ \hline
 \end{tabular}
\end{table}

Gummi és un editor molt més simple que Kile, amb previsualització en temps real però amb moltes menys funcionalitats. És bo per a documents simples però per a projectes complexos no. \\
%Taula 3
\begin{table}[h!]
 \begin{tabular}{|l|l|l|}
\hline
 \LaTeX & \textbf{Kile} & \textbf{Gummi} \\ \hline
 \textbf{complexibilitat:}  & X &  \\ \hline
 \textbf{Previsualització} &  & X \\ \hline
 \textbf{funcionalitat} & X &  \\ \hline
 \end{tabular}
\end{table}

Per més informació en qüestió de comparació podeu fer una consulta a les següents pagines: \href{https://osluca.uca.es/noticia/cinco-editores-de-latex-libres/}{5 editors de LaTeX} \href{https://latex.org/forum/viewtopic.php?t=208}{Analisí detallada de Kile i d'altres editors}
Fonts: \href{https://iloo.wordpress.com/2010/10/20/kile-otro-editor-latex/}{Experiencia Personal de l'us Kile} \href{https://www.kdeblog.com/editor-de-latex-para-kde-kile.html}{Descripcio de Característiques Kile}

\subsection{KDE}\label{subsec:KDE}
El nom "KDE" originalment sí que era un acrònim de "K Desktop Environment" (des de la seva creació el 1996), però a partir del 2009, la comunitat va decidir deixar de considerar-lo un acrònim i utilitzar-lo simplement com un nom propi per representar:
\begin{itemize}
 \item L'entorn d'escriptori
 \item La comunitat global que desenvolupa programari lliure
 \item Tots els projectes relacionats
\end{itemize}
Font:\href{https://ca.wikipedia.org/wiki/KDE}{Wikipedia}

\subsubsection{Programari lliure}

El programari (o software en anglès) és el conjunt de programes informàtics, procediments i documentació que permeten a un ordinador realitzar tasques específiques. És la part interna d'una computadora ,a diferencia del hardware que es l'extern, la part fisica.

Un programari es considera Free software quan respecta les quatres llibertats fonamentals definides per la Free Software Foundation (FSF):
\begin{enumerate}
 \item  \textbf{Llibertat 0:} Executar el programa amb qualsevol propòsit.
 \item  \textbf{Llibertat 1:} Estudiar i modificar el codi font (requereix accés al codi)
 \item  \textbf{Llibertat 2:} Redistribuir còpies per ajudar altres usuaris.
 \item  \textbf{Llibertat 3:} Millorar el programa i compartir les modificacions.
\end{enumerate}

Mapa conceptual del Free software:
\begin{center}
\hspace{-27mm}
 \includegraphics[scale=0.1]{./figures/mapa.png}
\end{center}

Font:\href{https://www.fsf.org/}{Free Software Foundation}

\subsubsection{L'entorn d'escriptori}
Un entorn d'escriptori en sistemes operatius és la interfície gràfica que permet als usuaris interactuar amb l'ordinador. Inclou elements com finestres, icones, panells, fons de pantalla i eines de gestió d'arxius.A continuació, mostrarem els principals entorns que n'hi ha:
\begin{enumerate}
 \item KDE Plasma
 \begin{itemize}
  \item \textbf{Característiques: } Altament personalitzable, amb temes globals, widgets i configuració avançada.
  \item \textbf{Recomanat per:} Usuaris que volen control total sobre l'aparença i el flux de treball.
 \end{itemize}

 \item GNOME
 \begin{itemize}
  \item \textbf{Característiques:} Disseny minimalista i enfocat en la productivitat. Compatible amb extensions per ampliar funcionalitats.
  \item \textbf{Recomanat per: }  Usuaris que prefereixen una experiència neta i senzilla.
 \end{itemize}


 \item XFCE
\begin{itemize}
  \item \textbf{Característiques:} Lleuger i ràpid, ideal per a maquinari antic o limitat.
  \item \textbf{Recomanat per: } Usuaris que busquen eficiència sense sacrificar funcionalitats bàsiques .
 \end{itemize}

 \item Cinnamon
\begin{itemize}
  \item \textbf{Característiques:} Interfície tradicional similar a Windows, amb personalització mitjana.
  \item \textbf{Recomanat per: } Usuaris que provenen de Windows i busquen una transició suau.
 \end{itemize}

\end{enumerate}


Fonts:\href{https://dev.to/xploitcore/kde-vs-gnome-vs-others-choosing-the-best-linux-desktop-environment-in-2025-ab5}{Dev} \href{https://planet.communia.org/index.php/en/node/77}{Planet}



\section{Entorn col·laboratiu}\label{sec:4.3}
A l'hora de crear un projecte en equip, pot arribar a ser complicat treballar amb 2 o més membres del grup a la vegada en els seus respectius dispositius si no s'escull una eina col·laborativa adequada, sobretot si s'està treballant amb molts fitxers.Pot haver problemes alhora de intercanviar o que sol pugui estar un membre traballant en un fitxer fa que sigui un absolutament caos.
\subsection{Google Drive}
Google Drive és una plataforma de Google que ofereix eines gratuïtes com un processador de text (Docs) i un entorn col·laboratiu. Tot i ser senzill i eficient, té algunes limitacions importants:
\begin{itemize}
 \item \textbf{Poc control en el procés col·laboratiu:} És difícil gestionar canvis simultanis o conflicts entre versions.
 \item \textbf{Limitacions en el control de versions:} No permet controlar detalladament els canvis que es fan i registrar.
 \item \textbf{Falta de privadesa:} Google té accés als teus arxius, cosa que pot ser un problema per a dades sensibles.
\end{itemize}
Aquestes limitacions fan que Google Drive no sigui l'opció ideal per el nostre TR la qual cosa ens fara escollir l'altre alternativa que teniem el \nameref{subsec:Git+GitHub+Vim}.
\subsection{Git+GitHub+Vim}\label{subsec:Git+GitHub+Vim}
Git és un sistema de control de versions que ofereix:
\begin{itemize}
 \item Historial dels canvis.
 \item Branques per desenvolupar diferents funcionalitat sense afectar la versió principal.
 \item Control total dels canvis
 \item Control i revisió de codi
\end{itemize}
El principal desavantatge que té el Git és que, per si sol, no està optimitzat per a la col·laboració en equips grans. Pero aquí entra en accio el GitHub, que complementa Git amb:
\begin{itemize}
\item \textbf{Repositoris}
 \item \textbf{Git pull,commit,push:} Dona la possibilitat de poder discutir les versions abans de pujarlo a la original.
 \item Solucions als conflictes de versions amb l'ajuda del \nameref{subsubsec:Vim}.
 \end{itemize}

\textbf{Avantatges:}

\begin{itemize}
 \item \textbf{Copia de seguretat:} Cada usuari té una copia de seguretat del servidor principal, en el cas d'errors o corrupció, sempre el pots reemplaçar-lo per el servidor principal.
 \item \textbf{Garantia de dades:} El model ded ades que utilitza Git garantitza la integritat ciptográfica de cada bit del teu projecte. Cada arxiu es recupera mitjançant la seva suma de comprobació quan es torna a desgarregar-se. Es imposible obtenir res de Git que no sigui els bits exactes que s'ha integrat.
 \item \textbf{Àrea de preparació:} Una cosa que diferència a Git d'altres eines es que es posible preparar alguns dels teus arxius i confirmar-los sense confirmar tots els altres canvis que s'han fet en els altres arxius, això et permet preparar nomès algunes parts d'un arxiu modificat i es realitza amb el commandament git commit ``nom de l'arxiu''. Git també et dona la opció d'ignorar aquesta característica i confirmar tots els canvis que s'han fet, afegint un ``-a'' al final del commandament.
 \item \textbf{Eficiència de Git:} Git no depèn d'internet per la majoria de les operacions, totes les accions principals(veure l'historial, fer commits) es fan en localment. Git està programat en llenguatge C, això permet a Git realitzar les tasques amb molta rapidesa encara que es treballi amb molts fichers.
\end{itemize}
Més informació en: \href{https://git-scm.com/about/branching-and-merging}
i

\subsubsection{Vim}\label{subsubsec:Vim}

\section{Llenguatge de programació}\label{sec:4.4}

El llenguatge de programació que vam decidir per fer la part pràctia del treball va ser Python. Python és un llenguatge de programació que va ser creat en 1989 per Guido van Rossum , un informàtic neerlandès. Aquest llenguatge és conegut per la seva simple sintaxi, comparada amb altres llenguatges com JavaScript o C++ que són més complicats, python facilita l'aprenentatge pels usuaris que començen a apendre a programar desde 0.

\subsection{Python}
La raó principal per la que vam escollir aquest llenguatgeera per les avantatges seguents:

\textbf{Avantatges}
\begin{itemize}
 \item \textbf{No cal compilar:} No és necesari comilar en Python, un cop que es té el codi, s'executa directament.
 \item \textbf{Llibreries} Python té una gran quantitat de llibreries molt útils com ``TensorFlow'' o numpy, aquestes llibreries estalvien moltes línies de codi a l'hora de programar una xarxa neuronal.
 \item \textbf{Identificació d'errors} Encara que altres llenguatges també poden identificar-te els erros, Python t'ho explica de manera més clara i precisa, estalviant-te el temps d'anar línea per línea buscant l'error.
 \item \textbf{Sintaxi:}La seva sintaxi clara i llegible que està explicada a l'apartat \ref{4.4.2} és ho que fa que aquest llenguatge sigui fàcil d'apendre quan un comença a programar desde 0.
\end{itemize}
Malgrat totes aquestes avantatges, Python tambè té unes porques desaventatges que hem de tenir present.

\textbf{Desavantatges:}
\begin{itemize}
 \item \textbf{Limitacions de rendiment:} La naturalesa interpretada de Python significa que és més lent que altres llenguatges. La causa d'això és que Python es tradueix a codi màquina línea per línea durant l'execució del codi, cosa que afegeix un temps de processament.
 \item \textbf{Consum de memoria:} Gràcies a la bona fexibilitat d'aquest llenguate fa que l'ordinador requereixi un major consum de memòria, un consum excesiu de la memoria pot causar relentizació en el programa i bloquejar-lo. Per tant es recomendable tenir un ordinador o portatil decent que pugui executar correctament Python.
\end{itemize}

\subsection{Sintaxi}
Acabem d'explicar que la sintaxi de Python és simple, pero què diferència aquesta sintaxi amb altres llenguatges? La principal raò és que el seu format de codi és visualment ordenat, utiliza paraules claus de l'anglès. A diferència d'altres llenguatges, no utilitza claudàtors per determinar blocs, com per exemple les sentències ``if, else''.

\begin{table}[h!]
\centering
\begin{tabular}{|p{7cm}|p{7cm}|}
\hline
\textbf{Condiciones en C} & \textbf{Condiciones en Python} \\ \hline

\begin{minipage}[t]{\linewidth}
\begin{lstlisting}[language=C]
#include <stdio.h>

int main() {
    int numero = 5;

    if (numero > 0) {
        printf("Es >0.\n");
    } else {
        printf("Ès <0.\n");
    }
    return 0;
}
\end{lstlisting}
\end{minipage}
&
\begin{minipage}[t]{\linewidth}
\begin{lstlisting}[language=Python]
numero = 5

if numero > 0:
    print("És positiu.")
else:
    print("És negatiu.")
\end{lstlisting}
\end{minipage}
\\ \hline
\end{tabular}
\caption{Comparació de condicions en C y Python}
\end{table}

En aquesta taula podem veurem la diferència de sintaxi que té cada llenguatge, en aquest cas podem apreciar que C és mès llarg que python i que requereis més commandaments i línies de codi. A continuació mostrarem una taula resumida d'aquestes diferències.

\textbf{Resum de diferències:}
\begin{table}[h!]
\centering
\begin{tabular}{|l|l|l|}
\hline
\textbf{Concepte} & \textbf{C} & \textbf{Python} \\
\hline
Terminador de línia & \verb|;| obligatori & No és obligatori \verb|;| \\
\hline
Agrupació de blocs & \{ \} & Espai en blanc \\
\hline
Condició & \verb|if (condició)| & \verb|if condició:| \\
\hline
Estructura & Requereix el script \verb|main()| & Script directe \\
\hline
Librerías & \verb|#include <stdio.h>| & No requereix per \verb|print()| \\
\hline
\end{tabular}
\caption{Resum de les diferències}
\end{table}













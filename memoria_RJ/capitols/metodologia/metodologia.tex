\chapter{Metodologia}
Després d'haver fet la recerca previa, i havent decidit fer una xarxa neuronal ``model'' que compleixi tots els requesits que esperavem,ara cal pensar amb quina metodologia podrem aconseguir el nostre objectiu.Llavors això vol dir que hem de fer un altre recerca en que utilizarem els millors eines per fer el Treball de Recerca de manera rapida i eficient.

Primer ens em fet la pregunta sobre quins ambits hem de recorre,una vegada feta la pregunta ja buscarem les millors eines o metode que ens ajudara en cadascun dels ambits.A continuació farem una presentacions dels ambits que vam tractar i les diferents solucions que vam donar i el definitiu.

Al començar tractem  l'apartat \ref{4.1}, que consisteix en com vam estructura la via comunicació entre nosaltres i el tutor.A l'apartat \ref{4.2},tracta de quin editor de text vam escollir,que es un punt crucial per el treball ja que moltes vegades les eines que ens aporta l'editor si mateix ens pot estalviar molt de temps i fer que el treball sigui visualment mes accessible.




\section{Comunicació}\label{4.1}
\section{Editor de text}\label{4.2}

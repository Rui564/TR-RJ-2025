\chapter{Metodologia}
Després d'haver fet la recerca previa, i havent decidit fer una xarxa neuronal ``model'' que compleixi tots els requesits que esperavem,ara cal pensar amb quina metodologia podrem aconseguir el nostre objectiu.Llavors això vol dir que hem de fer un altre recerca en que utilizarem els millors eines per fer el Treball de Recerca de manera rapida i eficient.

Primer ens em fet la pregunta sobre quins ambits hem de recorre,una vegada feta la pregunta ja buscarem les millors eines o metode que ens ajudara en cadascun dels ambits.A continuació farem una presentacio dels ambits que vam tractar i les diferents solucions que vam donar i el definitiu.

Al començar tractem  l'apartat \ref{4.1}, que consisteix en com vam estructura la via de comunicació entre nosaltres i el tutor.A l'apartat \ref{4.2},tracta de quin editor de text vam escollir,que es un punt crucial per el treball ja que  les eines que ens aporta l'editor per si mateix ens pot estalviar molt de temps i fer que el treball sigui visualment mes accessible.A continuació hi es l'apartat \ref{4.3},on t'explica la plataforma que hem escollit per l'entorn col·laboratiu.Finalment per poder crear una xarxa neuronal hem de utilizar un llenguatge de programació i això s'explicarà a l'apartat \ref{4.4}.




\section{Comunicació}\label{4.1}
\section{Editor de text}\label{4.2}
\section{Entorn col·laboratiu}\label{4.3}
\section{Llenguatge de programació}\label{4.4}


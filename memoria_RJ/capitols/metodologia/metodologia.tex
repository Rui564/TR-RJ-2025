\chapter{Metodologia}
Després d'haver fet la recerca previa, i d'haver decidit fer una xarxa neuronal ``model'' que compleixi tots els requesits que esperavem, ara cal pensar amb quina metodologia podrem aconseguir el nostre objectiu. Per tant vol dir que hem de fer un altre recerca en el que utilizarem les millors eines per fer el Treball de Recerca de manera ràpida i eficient.


Primer ens em fet la pregunta de  sobre quins àmbits hem de recorre, una vegada feta la pregunta ja buscarem les millors eines o mètode que ens ajudarà en cadascún dels àmbits. A continuació farem unes presentacions dels àmbits que vam tractar i les diferents solucions que vam donar i el definitiu.

Per començar tractem  l'apartat \ref{4.1}, que consisteix en com vam estructurar la via de comunicació entre nosaltres i el tutor. A l'apartat \ref{4.2},tracta de quin editor de text vam escollir, que es un punt crucial per el treball ja que les eines que ens aporta l'editor en si mateix ens pot estalviar molt de temps i convertirlo visualment més accessible. A continuació hi es l'apartat \ref{4.3}, on t'explica la plataforma que hem escollit per l'entorn col·laboratiu. Finalment per poder crear una xarxa neuronal hem de utilizar un llenguatge de programació i això s'explicarà a l'apartat \ref{4.4}.



\section{Comunicació}\label{4.1}
Com que som dos persones fent el treball amb un tutor,llavors és fonamental tenir un via comunicativa.Per tant,hem acollit diferents maneres de comunicació.Àra bé no em pogut registrar tots els registres que hem tingut però la gran majoria s'ha pogut conservar.\\
D'aquí cap avall explicarem les diferents maneres de comunicació utilizades en l'elaboracio d'aquest treball:\\
\subsection{Full de Calculs}
Hem realitzat diversios reunions amb el tutor,i la via que utilizem per configurar l'horari de la quedada es a traves d'un full de calcul.En aquestes quedades ens em  organitzat el treball i em quedat amb els diferents metodes que utilizarem per el treball.I el tutor ens arregla qualsevol tipos de problemes,i la instalació dels programes necessaris.
\subsection{Correu Electronica}
l'eina que utilizem a distancia es el servei de correu electronic Gmail.Vam fer l'us d'aquest aparell per el simple fet de que sigui simple pero eficient.I te una gran memoria d'emmagatzament per conservar totes les conserves que em feta.
Les comunicacions fetes per el correu electronic ha sigut la principal via de comunicacio que em establert,per la facilitat i l'eficiencia que ens aporta.Llavors tema ensenyament i dubtes simples o fem tot amb el gmail.

\subsection{Git}
També amb l'ajuda del git ens podem establir una comunicacio diaria amb el progres del treball deixant comentaris dels canvis i adaptacions que em fet.


\section{Editor de text i processador de text}\label{4.2}
Per aconseguir una bona presentació de TR,haviem d'escollir entre un editor de text o un processador.L'editor i el processador ambos son per crear documents pero cadascún te les seves funcions.\\
L'editor del text,es tal com el seu nom,serveix per editar els documents,es molt simple això genera inconvenients com no poder utilizar formats avançat o una personificacio.Pero, lo que te un editor es que tinguis la capacitat de navegar rapidament de fitxers entre fitxers,i facil per codificar en que te compabilitat amb tots tipos de llenguatge és lo ideal per a programadors.\\
Un processador de text,ja son tipos de programes més elaborades,més complexos.Sent programes amb tanta complexibilitat et dona l'àcces de utilizar formats avançat, utilitzat per a documents professionals com curriculum,llibres i d'altres més,també donen tot tipos d'opcions col·laboratives com disseny d epagina,correció grafica i molt més.

\section{Entorn col·laboratiu}\label{4.3}
\section{Llenguatge de programació}\label{4.4}


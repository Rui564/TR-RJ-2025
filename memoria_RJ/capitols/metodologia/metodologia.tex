\chapter{Metodologia}
Després d'haver fet la recerca previa, i d'haver decidit fer una xarxa neuronal ``model'' que compleixi tots els requesits que esperavem, ara cal pensar amb quina metodologia podrem aconseguir el nostre objectiu. Per tant vol dir que hem de fer un altre recerca en el que utilizarem les millors eines per fer el Treball de Recerca de manera ràpida i eficient.


Primer ens em fet la pregunta de  sobre quins àmbits hem de recorre, una vegada feta la pregunta ja buscarem les millors eines o mètode que ens ajudarà en cadascún dels àmbits. A continuació farem unes presentacions dels àmbits que vam tractar i les diferents solucions que vam donar i el definitiu.

Per començar tractem  l'apartat \ref{sec:4.1}, que consisteix en com vam estructurar la via de comunicació entre nosaltres i el tutor. A l'apartat \ref{sec:4.2}, tracta de quin editor de text vam escollir, que es un punt crucial per el treball ja que les eines que ens aporta l'editor en si mateix ens pot estalviar molt de temps i convertirlo visualment més accessible. A continuació hi es l'apartat \ref{sec:4.3}, on t'explica la plataforma que hem escollit per l'entorn col·laboratiu. Finalment per poder crear una xarxa neuronal hem d'utilizar un llenguatge de programació i això s'explicarà a l'apartat \ref{sec:4.4}.



\section{Comunicació}\label{sec:4.1}
Com que som dos persones fent el treball amb un tutor, ens és fonamental tenir una bona via comunicativa. Per tant, hem acollit diferents maneres de comunicació. Ara bé no em pogut registrar tots els registres que hem tingut però la gran majoria s'ha pogut conservar.\\
D'aquí cap avall explicarem les diferents maneres de comunicació utilizades en l'elaboracio d'aquest treball:\\
\subsection{Full de Calculs}
Hem realitzat diverses reunions amb el tutor, i la via que vam utilitzar per configurar l'horari de les quedades és a travès d'un full de càlcul. En aquestes quedades ens em organitzat el treball i ens em quedat amb els diferents mètodes que utilizariem per el trebal, el tutor ens arregla qualsevol tipus de problema, i la instalació dels programes necessaris.
\subsection{Correu Electronica}
L'eina que vam utilizar a distància és el servei de correu electronic Gmail. Vam fer ús d'aquest aparell per el simple fet de que és simple i eficient,i té una gran memòria d'emmagatzament per conservar totes les conserves que em fet.
Les converses fetes per el correu electrònic ha sigut la principal via de comunicació que em establert, per la facilitat i l'eficiència que ens aporta. Per tant, el tema ensenyament i dubtes simples o feiem tot per gmail.

\subsection{Git}
També amb l'ajuda del git, un software que ens va ajudar a guardar tots els canvis que hem anat fent durant el treball de recerca. Vam  establir una comunicació diària amb el progrés del treball, deixant comentaris dels canvis i adaptacions que vam anar fet.


\section{Editor de text i processador de text}\label{sec:4.2}
Per aconseguir una bona presentació de TR, haviem d'escollir o un editor de text o un processador. Tant l'editor com el processador són per crear documents, pero cadascún te les seves funcions.\\
L'editor del text, és tal com el seu nom, serveix per editar els documents. És molt simple i això genera inconvenients, com per exemple no poder utilizar formats avançat o una personificació. Però, les avantatge d'aquest és que pots tenir la capacitat de navegar rapidament de fitxer en fitxer, facilita la codificació i té compabilitat amb tots els tipus de llenguatge. És l'ideal per programadors.\\
Un processador de text són tipus de programes més elaborats i més complexos. Com que són programes amb tanta complexibilitat, et dona accès a formats avançat, utilitzat per a documents professionals com curriculums, llibres i d'altres més, també et donen tot tipus d'opcions col·laboratives com disseny de pàgina, correció gràfica i molt més.\\
Per a la part teòrica del nostre projecte sobre la Intel·ligència Artificial, vam seleccionar el format PDF degut a:
\begin{enumerate}
 \item Connexió segura i estable: Permet l'accés sense la dependència d'una connexió a Internet constant.
 \item Baixos requisits de recursos: Pot obrir-se en gairebé qualsevol dispositiu sense necessitat de programes específiques.
 \item Portabilitat: Manté el format independentment del sistema operatiu utilitzat.
 \item Multifuncionalitat: Admet text, fórmules matemàtiques, imatges i codi de programació integrat.
\end{enumerate}
L'estalvi de temps va ser un factor decisiu, ja que evita problemes de compatibilitat entre fórmules matemàtiques, imatges i el format del document.
En la nostra recerca d'eines que satisfacin tots els requisits, vam identificar dos tipus principals de processadors: Una que es deia WYSIWYG (What You See Is What You Get) i l'altre el WYSIWYM (What You See Is What You Mean). Ara donarem una explicació de les advantatges i desavantatges que tenen:
\subsection{What You See Is What You Get(WYSIWYG)}
WYSIWYG, es refereix a un tipus de processador que permet als usuaris veure en temps real el resultat final del document o disseny mentre editen, sense la necessitat de coneixer el codi o llenguatges de marcatges. Aquests tipus d'editors poden ser tant com el microsoft word o Google drive.

\textbf{Característiques}:
\begin{itemize}
 \item \textbf{Interfície visual:} Els canvis de format (negretes, taules, imatges) es fan mitjançant eines gràfiques, no codi.
 \item \textbf{Generació automàtica de codi:} El programa crea el codi subjacent (HTML, CSS, etc.) sense que l'usuari hi intervingui directament.
 \item \textbf{Ús generalitzat:} S'utilitza en àmbits com l'edició web, el disseny gràfic i la publicació digital.
\end{itemize}
Avantatges:
\begin{itemize}
\item \textbf{Edició visual intuïtiva:} Permet ajustar el format directament (arrossegar imatges, canviar fonts amb clics).
 \item Ideal per a usuaris sense coneixements tècnics.
\item \textbf{Resultat immediat:} No cal compilar, els canvis es veuen al moment.
\item \textbf{Eines integrades:} Funcions com correcció ortogràfica, taules gràfiques, o opcions de disseny accessibles des del menú
\item \textbf{Millor per a maquetació complexa:} Documents amb molts elements gràfics (pòsters, fulls informatius)
\end{itemize}
Desavantatges:
\begin{itemize}
 \item \textbf{Poca precisió en contingut tècnic:} Fórmules matemàtiques o codis es poden deformar-se a l'hora de canviar el format.
 \item \textbf{Errors en documents llargs:} El format manual pot provocar errors (salt de pàgina, numeració desorganitzada).
 \item \textbf{Dependència del programa:} Si s'obre en un altre software, el disseny pot variar.
 \item \textbf{Control de versions complicat:} Difícil treballar en equip sense conflictes de format.
\end{itemize}

\subsection{What You See Is What You Mean(WYSIWYM)}
WSYSIWYM, es refereix a un tipus de processador que es centra en l'estructura del contigut, no en la seva aparença visual immediata, l'usuari marca el text segons la seva funció (títol, secció, cita i d'altres més) i el format final s'aplica mitjançant un full d'estil com css o Latex.
Avantatges:
\begin{itemize}
\item \textbf{Precisió en elements tècnics:} Fórmules matemàtiques, codi o referències es gestionen amb sintaxi clara.
\item \textbf{Consistència automàtica:} L'estil s'aplica globalment.
\item \textbf{Lleuger i portable:} Els fitxers són de text pla.
 \item \textbf{Ideal per a treball en entorns col·laboratius:} Es pot actualizar amb git i fusionar sense haber de canviar el format.
\end{itemize}
Desavantatges:
\begin{itemize}
\item \textbf{Corba d'aprenentatge:} Cal aprendre una sintaxi específica.
 \item \textbf{Previsualització no immediata:} En LaTeX, cal compilar; en Markdown, cal un renderitzador.
 \item \textbf{Limitacions en disseny gràfic}: Personalització avançada (ex: posicionament exacte d'imatges) requereix codi addicional.
\end{itemize}

\begin{table}[h!]
\begin{tabular}{|l|l|l|}
\hline
\textbf{Aspecte} & \textbf{WYSIWYG} & \textbf{WYSIWYM} \\ \hline
\textbf{Enfocament} & Aparença visual immediata & Estructura semàntica del contingut \\ \hline
\textbf{Exemples} & Microsoft Word, WordPress & LaTeX, LyX \\ \hline
\textbf{Usuaris} & No tècnics, dissenyadors &  Acadèmics, desenvolupadors tècnics \\ \hline
\textbf{Control} &  Limitada (codi generat automàtic) & Alt (definició manual de l'estructura) \\ \hline
\end{tabular}
\caption{Comparativa entre WYSIWYG i WYSIWYM}
\end{table}



font(WYSIWYG): (\href{https://es.wikipedia.org/wiki/WYSIWYG}{Wikipedia} \href{https://es.wikipedia.org/wiki/WYSIWYG}{Arimetrics} \href{https://www.wix.com/encyclopedia/definition/wysiwyg}{Wix}).
font(WYSIWYM): (\href{https://en.wikipedia.org/wiki/WYSIWYM}{Wikipedia} \href{https://en.ryte.com/wiki/WYSIWYG/}{Ryte WIKI}).



\section{\LaTeX}
\begin{figure}[h]
    \centering
    \includegraphics[width=0.5\textwidth]{Latex.png}
\end{figure}



























\section{Entorn col·laboratiu}\label{sec:4.3}
\section{Llenguatge de programació}\label{sec:4.4}
El llenguatge de programació que vam decidir per fer la part pràctia del treball va ser Python. Python és un llenguatge de programació que va ser creat per Guido van Rossum, un informàtic neerlandès. Aquest llenguatge és conegut per la seva simple sintaxi, comparada amb altres llenguatges com JavaScript o C++, python facilita l'aprenentatge pels usuaris que començen a apendre a programar. Aquesta va ser la principal raó per la que vam escllir python com a llenguatge de programació en la part pràctica del nostre treball, pero encara hi ha més.

\subsection{Perquè vam escollir Python?}
Voliem reduïr al màxim el nostre temps de trevall per estar més còmodes i no estar molt apresurats amb el temps. Les avantatges d'utilitzar python són:
\begin{itemize}
 \item Facilitat d'apendre un llenguatge desde 0
 \item Té múltiples llibreries gratuites com ``TensorFlow'' o numpy, aquestes llibreries estalen molts línies de codi a l'hora de programar una xarxa neuronal, per tant també ens estalvia temps.
 \item És multiplataforma, el podem descarregar pràcticament en qualsevol sistema operatiu (Windows, Linux, Unix)
 \item Et pot identificar els errors i sugerir-te millores.
\end{itemize}





\chapter{Metodologia}
Després d'haver fet la recerca previa, i d'haver decidit fer una xarxa neuronal ``model'' que compleixi tots els requesits que esperavem, ara cal pensar amb quina metodologia podrem aconseguir el nostre objectiu. Per tant vol dir que hem de fer un altre recerca en el que utilizarem les millors eines per fer el Treball de Recerca de manera ràpida i eficient.


Primer ens em fet la pregunta de  sobre quins àmbits hem de recorre, una vegada feta la pregunta ja buscarem les millors eines o mètode que ens ajudarà en cadascún dels àmbits. A continuació farem unes presentacions dels àmbits que vam tractar i les diferents solucions que vam donar i el definitiu.

Per començar tractem  l'apartat \ref{4.1}, que consisteix en com vam estructurar la via de comunicació entre nosaltres i el tutor. A l'apartat \ref{4.2}, tracta de quin editor de text vam escollir, que es un punt crucial per el treball ja que les eines que ens aporta l'editor en si mateix ens pot estalviar molt de temps i convertirlo visualment més accessible. A continuació hi es l'apartat \ref{4.3}, on t'explica la plataforma que hem escollit per l'entorn col·laboratiu. Finalment per poder crear una xarxa neuronal hem d'utilizar un llenguatge de programació i això s'explicarà a l'apartat \ref{4.4}.



\section{Comunicació}\label{4.1}
Com que som dos persones fent el treball amb un tutor, ens és fonamental tenir una bona via comunicativa. Per tant, hem acollit diferents maneres de comunicació. Ara bé no em pogut registrar tots els registres que hem tingut però la gran majoria s'ha pogut conservar.\\
D'aquí cap avall explicarem les diferents maneres de comunicació utilizades en l'elaboracio d'aquest treball:\\
\subsection{Full de Calculs}
Hem realitzat diverses reunions amb el tutor, i la via que vam utilitzar per configurar l'horari de les quedades és a travès d'un full de càlcul. En aquestes quedades ens em organitzat el treball i ens em quedat amb els diferents mètodes que utilizariem per el trebal, el tutor ens arregla qualsevol tipus de problema, i la instalació dels programes necessaris.
\subsection{Correu Electronica}
L'eina que vam utilizar a distància és el servei de correu electronic Gmail. Vam fer ús d'aquest aparell per el simple fet de que és simple i eficient,i té una gran memòria d'emmagatzament per conservar totes les conserves que em fet.
Les converses fetes per el correu electrònic ha sigut la principal via de comunicació que em establert, per la facilitat i l'eficiència que ens aporta. Per tant, el tema ensenyament i dubtes simples o feiem tot per gmail.

\subsection{Git}
També amb l'ajuda del git, un software que ens va ajudar a guardar tots els canvis que hem anat fent durant el treball de recerca. Vam  establir una comunicació diària amb el progrés del treball, deixant comentaris dels canvis i adaptacions que vam anar fet.


\section{Editor de text i processador de text}\label{4.2}
Per aconseguir una bona presentació de TR, haviem d'escollir o un editor de text o un processador. Tant l'editor com el processador són per crear documents, pero cadascún te les seves funcions.\\
L'editor del text, és tal com el seu nom, serveix per editar els documents. És molt simple i això genera inconvenients, com per exemple no poder utilizar formats avançat o una personificació. Però, les avantatge d'aquest és que pots tenir la capacitat de navegar rapidament de fitxer en fitxer, facilita la codificació i té compabilitat amb tots els tipus de llenguatge. És ho ideal per programadors.\\
Un processador de text són tipus de programes més elaborats i més complexos. Com que són programes amb tanta complexibilitat, et dona l'àcces d'utilizar formats avançat, utilitzat per a documents professionals com curriculums, llibres i d'altres més, també et donen tot tipus d'opcions col·laboratives com disseny de pàgina, correció gràfica i molt més.

\section{Entorn col·laboratiu}\label{4.3}
\section{Llenguatge de programació}\label{4.4}



\chapter{Answer}




\newtcolorbox{modernquote}{
    colback=gray!10,
    colframe=white,
    boxrule=0pt,
    leftrule=4pt,
    colupper=black,
    arc=0pt,
    outer arc=0pt,
    left=8pt,
    right=8pt,
    top=8pt,
    bottom=8pt,
    before skip=12pt,
    after skip=12pt,
    enhanced jigsaw,
    borderline west={4pt}{0pt}{gray!60}
}
\begin{modernquote}
    Dwar Ev va soldar cerimoniosament la connexió final amb or. Els ulls d’una dotzena de càmeres de televisió l’observaven i el subeter transmetia a tot l’univers una dotzena d’imatges del que ell feia.

Es va redreçar i va assentir amb el cap a Dwar Reyn, després es va col·locar al costat de l’interruptor que completaria el contacte quan el llancés. L’interruptor que connectaria, tot d’una, totes les monstruoses màquines de computació de tots els planetes habitats de l’univers —noranta-sis mil milions de planetes— en el supercircuit que les enllaçaria totes en una sola supercalculadora, una màquina cibernètica que combinaria tot el coneixement de totes les galàxies.

Dwar Reyn va parlar breument als bilions d’oients i espectadors. Després d’un moment de silenci, va dir:

—Ara, Dwar Ev.


Dwar Ev va activar l’interruptor. Hi va haver un brunzit poderós, l’alliberament d’energia procedent de noranta-sis mil milions de planetes. Les llums van centellejar i després es van apagar al llarg del tauler de quilòmetres de llargada.

Dwar Ev es va fer enrere i va respirar profundament.

—L’honor de fer la primera pregunta és teu, Dwar Reyn.



—Gràcies —va dir Dwar Reyn—. Serà una pregunta que cap màquina cibernètica per si sola no ha estat capaç de respondre.


Es va girar per mirar la màquina.

—Hi ha un Déu?


La veu poderosa va respondre sense vacil·lar, sense que fes clic cap relé:

—Sí, ara hi ha un Déu.


Una por sobtada va aparèixer al rostre de Dwar Ev. Va saltar per agafar l’interruptor.

Un llamp, vingut d’un cel sense núvols, el va fulminar i va fondre l’interruptor tancat.


    \hfill\textit{-- Autor desconegut}
\end{modernquote}



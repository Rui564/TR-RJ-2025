\chapter{Resum}
\label{c:Resum}
\section{Resum}
El 1956, un petit grup de científics es va reunir a Dartmouth College amb una idea que semblava de ciència-ficció, crear màquines capaces de pensar. Aquella idea va ser l'inici del qual avui coneixem com a intel·ligència artificial, un camp que impulsa el motor del progrés tecnològic a la societat actual. En unes dècades, la IA ha passat de ser una idea impensable a convertir-se en una eina que ens envolta diàriament, des dels assistents virtuals fins als sistemes de diagnòstic mèdic o els vehicles autònoms.

Dins aquest univers, destaquen les xarxes neuronals artificials, inspirades en el funcionament del cervell humà. Aquestes estructures matemàtiques i computacionals han demostrat una capacitat extraordinària per aprendre de les dades, reconèixer patrons i resoldre problemes complexos. La seva influència creix de manera exponencial, generant avenços impressionants.

En aquest treball s'explorarà el món de la intel·ligència artificial i, especialment, el de les xarxes neuronals. Se n'explicaran els fonaments i les aplicacions en diferents àmbits, tant en l'àmbit tècnic com ètic.

\section{Abstract}
In 1956, a small group of scientists gathered at Dartmouth College with an idea that seemed like science fiction: creating machines capable of thinking. That idea marked the beginning of what we now know as artificial intelligence, a field that drives the engine of technological progress in today’s society. In just a few decades, AI has gone from being an unthinkable concept to becoming a tool that surrounds us daily, from virtual assistants to medical diagnostic systems or autonomous vehicles.

Within this universe, artificial neural networks stand out, inspired by the functioning of the human brain. These mathematical and computational structures have demonstrated an extraordinary ability to learn from data, recognize patterns, and solve complex problems. Their influence is growing exponentially, leading to impressive advances.

This paper will explore the world of artificial intelligence, and especially that of neural networks. It will explain their foundations and applications in different fields, both from a technical and an ethical perspective.

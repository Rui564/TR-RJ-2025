\chapter{Conclusions}
\label{c:conclusions}
En aquest capitol mostrarem tots els resultats que em aconseguit a traves dels objectius:
\begin{enumerate}
     \item \textbf{Entorn de treball professional: }
     \item \textbf{Escalable: }
     \item \textbf{Simplicitat i Eficiència: }
     \item \textbf{Llibertat: }
     \item \textbf{Xarxa neuronal: }
     \item \textbf{Full de càlcul}
\end{enumerate}

\section{Comparació entre una xarxa neurnal creada per un llenguatge de programació netre una de fulls de calculs}
En aquest secció, desprès de que cadascún de nosaltras haguèssim acabat les nostres respectives pràctiques, compararem els nostres resultats finals i veurem quina de les 2 formes és millor.\\

La principal diferencia que destaca entre aquests dues metodes de desenvolupament es la automatizació de l'aprenentatge (\textbf{Automatic machine}. En el cas dels fulls de càlcul, cal optimitzar i ajustar-ho tot manualment, cosa que esdevé un procés molt laboriós. En canvi, amb el llenguatge de programació, la XNA te'l fa tot el procés manual. Tanmateix, els fulls de calculs ho té tot més visual, permetent a l'usuari veure tot el treball, però en Python està oculta.\\

Per altra banda, la precisió de la predicció també és un factor molt important. En el nostre cas, amb un full de càlcul s’han predit 15 enquestes, amb una precisió del 66,67\%. En canvi, amb Python s’ha obtingut una precisió del 94,44\%. Com podem observar la precisió del Python és clarament superior a la d'un full de càlcul. A més, Python ha predit més notes de matemàtiques.\\

Un altre factor complementari és la corba d'aprenentatge de cadascú. En aquest sentit per utilitzar Python t'hauries d'aprendre el llenguatge i les seves lògiques; per contra, un full de càlcul pot ser utilitzat fàcilment per qualsevol persona que hagi cursat estadística.\\

En conclusió, tot i que els fulls de càlcul ofereixen una major simplicitat i visualització del procés, el llenguatge de programació (Python) destaca per la seva capacitat d’automatització, eficiència i precisió en els resultats. Per tant, els fulls de càlcul són útils per a tasques bàsiques i didàctiques, però quan es busca rendiment i fiabilitat, l’ús de Python és clarament superior.

\chapter{Conclusions}

\label{c:conclusions}

Després dels resultats obtinguts, podem concloure que el treball està completat satisfactòriament. En aquest capítol presentem els assoliments dels objectius inicialment plantejats.

\begin{enumerate}

     \item \textbf{Entorn de treball professional: } S’ha aconseguit reproduir de manera notable el funcionament dels centres d’investigació en xarxes neuronals, amb l'ajuda de Linux, Git, GitHub i d'altres eines.

     \item \textbf{Escalabilitat: } La xarxa neuronal ha complert amb èxit aquest requisit, ja que és fàcilment ampliable mitjançant modificacions de paràmetres, incorporació de noves dades o augment de la seva complexitat.

     \item \textbf{Llibertat: } La publicació del projecte a GitHub ha permès convertir-lo en programari lliure, convertint-lo accessible, modificable i millorables per qualsevol persona.

     \item \textbf{Simplicitat i eficiència: } El model s’ha simplificat al màxim sense perdre eficàcia, assolint un equilibri entre senzillesa i rendiment.

     \item \textbf{Traçabilitat: } La publicació del projecte a GitHub dota al projecte de traçabilitat. El nostre repositori permet fer un seguiment pas per pas de l'elaboració del treball de recerca.

     \item \textbf{Xarxa neuronal: } En conclusió, tal com s’ha exposat en l’apartat comparatiu, la xarxa neuronal és superior al full de càlcul. Tanmateix, per a models simples continua sent recomanable l’ús del full de càlcul.

\end{enumerate}

Tot i que el treball ha estat dur i que ens ha calgut força esforç en alguns moments, l’experiència ha estat majoritàriament positiva. Hem gaudit molt en l’elaboració d’aquest projecte i hem tingut una gran motivació al llarg del procés. Gràcies a aquesta experiència hem pogut adquirir coneixements que mai ens havíem plantejat, cosa que ens permetrà tenir un cert avantatge en els inicis dels estudis universitaris. A més, aquest treball ens ha ajudat a millorar en la redacció, que fins ara havia estat un dels nostres punts febles.

\section{Recerca futura}
En el camp de la recerca, el final d'un treball pot ser l'inici d'un nou projecte de recerca. Per això creiem convenient fer propostes de com es podria continuar la nostra recerca.
% De cara al futur, hem definit una sèrie de possibles ampliacions que considerem beneficioses per a la nostra xarxa neuronal:


\begin{itemize}

\item Ampliar les sortides de la xarxa neuronal, ja que actualment només retorna un únic valor.

\item Millorar la velocitat de resposta, perquè en alguns casos el temps d’espera per obtenir un resultat precís era excessiu.

\item Desenvolupar nous models de la xarxa neuronal aplicats a casos reals, com el reconeixement de dígits, d’imatges o d’altres àmbits que hem considerat especialment interessants. Tanmateix, les limitacions en el temps i en l’extensió del treball no ens han permès dur-los a terme.

\item Crear una plataforma amb HTML\cite{HTML} i CSS\cite{CSS}, on poder publicar les noves actualitzacions i oferir guies a la comunitat per a principiants, tal com nosaltres ho érem en començar el TR.

\end{itemize}


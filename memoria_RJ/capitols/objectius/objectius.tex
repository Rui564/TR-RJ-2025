\chapter{Objectius}
\label{c:objectius}
Tal com podem veure en la \nameref{c:intro}, el nostre objectiu del TR (Treball de Recerca) és construir una xarxa neuronal, però no ens val qualsevol xarxa neuronal, ja que n'hi han d'infinites  models, per dur a terme aquest estudi, vam seleccionar tres models de xarxes neuronals diferents: una implementada en Python, una altra basada en fulls de càlcul i una tercera aplicada a un exemple real del joc Mobile Legends: Bang Bang. A partir d'aquí desglossem 4 apartats amb diferents objectius: \nameref{sec:Xarxa neuronal amb llenguatge de programació} \nameref{sec:Xarxa neuronal amb fulls de calculs} \nameref{sec:Xarxa neuronal amb un cas real} \nameref{sec:Objectius personals}


\section{Xarxa neuronal amb llenguatge de programació}\label{sec:Xarxa neuronal amb llenguatge de programació}
\section{Xarxa neuronal amb fulls de calculs}\label{sec:Xarxa neuronal amb fulls de calculs}
\section{Xarxa neuronal amb un cas real}\label{sec:Xarxa neuronal amb un cas real}
\section{Objectius personals}\label{sec:Objectius personals}


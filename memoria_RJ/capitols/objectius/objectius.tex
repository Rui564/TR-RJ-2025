\chapter{Objectius}
\label{c:objectius}
Un cop plantejat el nostre tema del Treball de recerca, ens hem establert uns objectius a seguir.

En aquest capítol parlarem dels objectius que ens hem proposat abans de començar el treball de recerca. Aquesta part la desglosarem en 3 parts diferens, per  una banda els nostres objectius pràctics, que és on explicarem ho que volem crear,  desprès els objectius teòrics, on explicarem tota la part teòrica del tema, i finalment els nostres objectius personals.

\section{Objectius teòrics}
Els objectius que volem assolir de la part teòrica en aquest treball són els seguents:

\begin{enumerate}
 \item Apendre el funcionament de la intel·ligència artificial.

 \item Conèixer com funcionen les xarxes neuronals que forma una intel·ligència artificial i la seva estructura.

\item Identificar i saber els diferents tipus d'algoritmes que té la intel·ligència artificial.

\item Saber com aplicar a la pràctica les xarxes neuronals.

\end{enumerate}

\section{Objectius pràctics}
Els nostres objectius pràctics són les següents:

\begin{enumerate}
 \item Programar una xarxa neuronal en llenguatge Python que pugui identificar una xifra del 1 fins al 9 i que aquest sigui capaç d'apendre dels seus errors.

\end{enumerate}

\section{Objectius personals}
Deixant a part dels objectius propis del treball de recerca ens hem proposat nosaltres mateixos uns objectius personals en aquest treball.
\begin{enumerate}
 \item Apendre a utilitzar LaTeX com a sistema de composició de textos.

 \item Apendre a programar en llenguatge Python

 \item Apendre a utilitzar Linux com a sistema operatiu.

 \item Apendre a utilitzar git com a entorn col·laboratiu i Vim com a editor de text.

\end{enumerate}


\chapter{Objectius}
\label{c:objectius}
Tal com podem veure en la \nameref{c:intro}, el nostre objectiu del TR (Treball de Recerca) és construir una xarxa neuronal, però no ens val qualsevol xarxa neuronal, ja que n'hi han d'infinites  models, per dur a terme aquest estudi, vam seleccionar tres models de xarxes neuronals diferents: una implementada en Python, una altra basada en fulls de càlcul i una tercera aplicada a un exemple real del joc Mobile Legends: Bang Bang. A partir d'aquí desglossem 4 apartats amb diferents objectius: \nameref{sec:Xarxa neuronal amb llenguatge de programació} \nameref{sec:Xarxa neuronal amb fulls de calculs} \nameref{sec:Xarxa neuronal amb un cas real} \nameref{sec:Objectius personals}


\section{Xarxa neuronal amb llenguatge de programació}\label{sec:Xarxa neuronal amb llenguatge de programació}
En aquest apartat fixem els objectius, o més ben dit requisits de la xarxa neuronal que farem amb llenguatge de programació.
\begin{enumerate}[label=\alph*)]
 \item Fer ús de Python com a llenguatge principal
 \item Utilitzar llibreries bàsiques com \texttt{NumPy} i \texttt{TensorFlow} per facilitar els càlculs
 \item Crear una xarxa capaç de reconèixer patrons senzills (per exemple, classificació de dades)
 \item Documentar el procés de disseny i d’implementació del codi
\end{enumerate}


\section{Xarxa neuronal amb fulls de calculs}\label{sec:Xarxa neuronal amb fulls de calculs}
Els objectius d’aquest apartat són:
\begin{enumerate}[label=\alph*)]
\item Implementar manualment els càlculs bàsics d’una xarxa neuronal en un full de càlcul (propagació cap endavant i retropropagació)
\item Mostrar de forma visual com funcionen les operacions matemàtiques internes
\item Comparar l’eficiència i la dificultat respecte a la implementació en Python
\end{enumerate}


\section{Xarxa neuronal amb un cas real}\label{sec:Xarxa neuronal amb un cas real}
Aquí volem aplicar els coneixements anteriors a un context més proper i pràctic:
\begin{enumerate}[label=\alph*)]
\item Escollir un cas concret relacionat amb el joc \textit{Mobile Legends: Bang Bang}
\item Recrear el cas real e intentar apropar-nos al màxim a la seva lògica
\item Comunicar-nos amb els treballadors del joc com a font de recursos
\end{enumerate}


\section{Objectius personals}\label{sec:Objectius personals}
A més dels objectius tècnics, també hi ha objectius d’aprenentatge personal:
\begin{enumerate}[label=\alph*)]
\item Dominar les funcionalitats basiques del \LaTeX, Vim, Github i Git
\item Millorar el nostre forma de redactar
\item enriquir de coneixements i gaudir del treball
\end{enumerate}




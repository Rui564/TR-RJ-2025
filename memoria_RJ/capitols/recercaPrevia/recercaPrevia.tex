 \chapter{Recerca prèvia}
\label{c:intro}
\section{La Història de la Intel·ligència Artificial: Des dels Orígens fins Avui}
\subsection{El Naixement d'una Idea Revo lucionaria (1956)}
Tot va començar amb una pregunta provocadora d’Alan Turing, el geni matemàtic que va desxifrar Enigma, una màquina emprada pels nazis per codificar els seus missatges durant la Segona Guerra Mundial (1939-1945): "Podran les màquines pensar alguna vegada?". Aquesta qüestió va obrir les portes a un nou camp d’estudi. En 1956 John McCarthy, Marvin Minsky i d'altres especialistes van nominar oficialmenta el terme ``Inteligencia artificial'' durant la famosa conferencia de Darmouth, marcant el inici d'una nova era tecnologica.
\subsection{El Joc que ho va canviar tot: The Imitation Game/El test de Turing}
El nucli de la IA es basa en un experiment molt senzill però profund:El joc d'imitació(The imitation game), proposat per Alan Turing. Davant de la pregunta "Podran les màquines pensar alguna vegada?", Turing va dissenyar un joc que funcionava com a test per les màquines anomenat ``The Imitation Game''. Aquest test consistia en que un avaluador havia de començar una conversa en forma de textos escrits amb una persona i una máquina durant 5 minuts, aquest avaluador no sabia qui era qui i el seu objectiu era esbrinar qui era l'humà. Si la màquina aconseguia enganyar a l'avaluador pasaba el test i es reconeixia que la màquina havia aconseguit un nivell de comportament lingüístic equivalent a la d'un humà, i donava resposta a la pregunta d'Alan Turing. Al cap dels anys, el joc ha estat evolucionant i moderat per els prodigis de la humanitat fins que avui en dia es conegut com el test de Turing. Tot això va ser clau en  donar lloc al naixement de la IA i per l'avanç de la tecnologia.
\subsection{ELs grans fites de la IA}
\subsubsection{1997: La maquina que va vencer un campio}
 La supercomputadora Deep Blue desnvolupada per IBM va derrotar el campió mundial d’escacs, Garry Kasparov, demostrant que la IA podia superar els humans en jocs d’estratègia complexos.
\subsubsection{2022: L'explosió de la IA}
Milions d’usuaris van descobrir models com ChatGPT que podien escriure, traduir i programar amb un llenguatge gairebé humà, obrint nous horitzons en la interacció home-màquina.
\subsubsection{2025: La IA en tots els ambits}
    2025: La IA en Tots els Àmbits
    Avui, la IA està present en dibuix, contingut audiovisual, cotxes autònoms, medicina i molt més, amb models cada vegada més especialitzats i avançats.



(Fonts: \cite{McCarthy_Minsky_Rochester_Shannon_2006}, \cite{deep-blue},
\href{https://openai.com/index/chatgpt/}{ChatGPT (2022)},
\cite{10.1093/mind/LIX.236.433}
)
\section{Historia de les xarxes neuronals artificials}
La història de les xarxes neuronals és molt extensa, per tant, resumirem molt aquest apartat, parlant molt breument de les creacions de xarxes més importants.
\subsection{Perceptron (1958)}
En la dècada de 1950 a 1960 el psicòleg i informàtic Frank Rosenblatt va crear el Percecptrò, la primera xarxa neuronal creada, a partir d'aquest moment, es potenciarien les xarxes neuronals.

Aquest model pren varies entrades binaries $x_1$, $x_2$, fins les que calguin, i produeix una sola sortida binaria. Per calcular la sortida, Rosenblatt va introduir els ``pesos'', que está explict a l'apartat \ref{sec:3.8}, els seus principals usos són decsions binàries senziles, o per crear funcions lògiques com OR o AND.

\subsection{Multiplayer Perceptron}
El multiplayer perceptron és una ampliació de la percepció d'una única neurona a més d'una. A més, apareix el concepte de capes d'entrades, capes ocultes i capes de sortida, (tot això explicat a l'apartat \ref{sec:3.6}) però amb valors d'entrada i rtida binàries. No hem d'ignorar que els científics asignaban el valor del pes i del umbral manualment en cada neurona, quan més perceptrons havien en les capes, era molt més difícil establir els pesos desitjats.

\subsection{Neurones sigmoide}
Per aconseguir que les xarxes neuronals aprenguin per elles mateixes, es a dir, aprenentatge automàtic \ref{3.4.2}, va ser necessari introduir un nou tipus de neurones, que són les Neurones Sigmoides, que són similar al perceptró, aquestes neurones en comptes de que les entrades siguin 1 o 0, puguin tenir valors com 0.5, o 0.374 o qualsevol altre valor real. Ara les sorides en lloc de ser 0 o 1, serà d(w . x + b), on d serà la funció sigmoide, explicat en l'apartat \ref{3.7.1}. Aquesta va ser la primera funció d'activació \ref{3.5.3}.

\subsection{Xarxa neuronal prealmentada (Feedforward)}
Les xarxes neuronals prealimentades són les que les sortides d'una sola capa són utilitzades com entrades en la pròxima, com explicarem més endevant en l'apartat \ref{sec:3.6} capa, on les conexions entre les unitats no dornen un cicle.

\section{Que és la Ia?}
Hem estat parlant molt durant aquest treball sobre la IA, i ara que em après quina és la seva història, hem de saber que és. Doncs bé, podem definir la IA com sistemes de software i de hardware disenyats per humans que actúen en un la dimensió física o digital, es a dir, raonar sobre el coneixemen, processant la informació derivada de dades i prendre les millors decisions per assolir l'objectiu donat. O dit d'un altre manera, és un camp de la informàtica que consisteix en un conjunt de capacitats interectuals i cognitives expresades per un sistema informàtic creat pels humans que té com a proposit imitar la intel·ligència humana, com escriure poemes, reconeixer imatges, fer prediccions basades en dades i més funcions. Un exemple de IA i que tot el món coneix i utilitza és el ChatGPT, un chatbot impulsada per un model d'intel·ligència artificial generativa de la empresa OpenAI. Utilitza técniques de processament de llenguatge natural per comprendre preguntes fetes per l'usuari i generar respostes coherents en converses, simulant una interacció similar a la d'un humà. Aquesta IA ha avançat molt desde el seu llençament, ara pot generar o editar imatges, procesar audios, llegir arxius, comrendre imatges i molt més.

\section{Com funciona la IA?}
Una vegada que ja sabem que és una IA, ens toca entendre com funciona. Les intel·ligències arficials utilitzen algoritmes i models matemàtics per processar grans quantitats de dades i prendre accions basades en patrons i regles establertes a travès de l'aprenentatge automàtic o l'aprenentatge profund. Per tant per funcionar necessitara:  \nameref{subsec:Dades}  \ \nameref{subsec:Algorismes}  \ \nameref{subsec:Potència computacional} \ \nameref{subsec:Software i Frameworks} \ \nameref{subsec:Optimització i Ajustos}  \ \nameref{subsec:Aplicació pràctica} \ \nameref{subsec:Ètica i Regulació}

\subsection{Dades}\label{subsec:Dades}
Les dades son fonamentals per la IA ja que es la base de l'aprennetatge del model, per poder raonar, prendre decisions, i millorar la presició. Aquí esdevenen uns exemples que pot haver:

\subsubsection{Basé d'aprenentage}
\begin{itemize}
 \item Les algoritmes de la IA necessiten a base de dades i una gran diversitat de dades per poder identificar patrons i construir prediccion.
\end{itemize}
\subsubsection{Qualitat vs Quantitat}
\begin{itemize}
\item Una gran quantitat de dades ajudaran a la IA a obtendendre major precisió, però la qualitat es encara més important per la complexibilitat de dades que aporta, això evitara que la IA cometes errors per informacio incompleta.\textbf{ Per exemple:} {\color{blue}En l'ambit medic si vols que la IA fagi una predicció i tan sol li dones una quantitat important de persones sanes i no d'altres exemplars, la IA simplement descartara d'altres possibilitats que podrian haver i només agafar la sana.}
\end{itemize}
\subsubsection{Exemples Reals}
\begin{itemize}
 \item Les sistemes dels cotxes o assistents virtuals necessiten dades en temps reals per adaptar-se de l'entorn, també plataformes com Netflix o Spotify necessiten dades personalitzades per poder generar recomanacions amb presició.
\end{itemize}
\subsubsection{Legisme i Etica}
\begin{itemize}
 \item A la IA s'ha d'aplicar dades com protecció de dades per favorir l'etica i moral, i se li ha d'entrenar amb fonts legitima, valides per tal d'evitar erros legals o tecnics
\end{itemize}


\subsection{Algorismes}\label{subsec:Algorismes}
\subsection{Algoritme Gredient descendent:}
\subsubsection{L'aprenentage automàtic(Machine learning)}
\begin{itemize}
 \item L'aprenentatge automàtic és una branca crucial de la intel·ligencia artificial, consisteix en cobrar vida a la maquina, donar-li el poder d'aprenentatge dels humans, realitzar tasques de manera autonoma i finalment les infinites possibilitats de evolucionar a traves de l'experiencia i molt més dades. Segons la \href{https://ischoolonline.berkeley.edu/blog/what-is-machine-learning/}{UC Berkeley} el procéss automatic es pot dividir-se en tres parts:

  \begin{enumerate}
  {\color{gray}
   \item \textbf{Mecanisme de predicció}
    \subitem\hspace*{-1\leftmargin} Un conjunt de regles o operacions matemàtiques que analitza les dades d'entrada i intenta identificar els patrons que busca el model.
   \item \textbf{Funció de pèrdua (o error)}
   \subitem\hspace*{-1\leftmargin} Un sistema per avaluar l'encert de les prediccions, comparant-les amb resultats reals (quan es disposa d'ells). Si la predicció és incorrecta, aquesta funció quantifica la magnitud de l'error.
   \item \textbf{Algorisme d'optimització}
   \subitem\hspace*{-1\leftmargin} El procés que ajusta automàticament el model per minimitzar l'error, modificant els paràmetres interns per millorar les prediccions futures.} \ \footnote{Llengua original: Anglessa; Traduit per el chat bot Deepseek}
  \end{enumerate}
   \end{itemize}
  Segons \href{https://blogs.nvidia.com/blog/supervised-unsupervised-learning/}{Nvidia} hi han molts tipus d'apranentage automatics:
  \begin{description}
   \item \nameref{subsubsec:Aprenentatge supervisat}
   \item \nameref{subsubsec:Aprenentatge semi-supervisat}
   \item \nameref{subsubsec:Aprenentatge no supervisat}
   \item \nameref{subsubsec:Aprenentatge reforç}
  \end{description}
\subsubsection{Aprenentatge supervisat}\label{subsubsec:Aprenentatge supervisat}
L'aprenentatge supervisat és un tipus d'aprenentatge automàtic que treballa amb dades etiquetades, és a dir, dades que ja inclouen la solució o resultat desitjat. En aquest mètode, la intel·ligència artificial aprén a associar les dades d'entrada amb les seves etiquetes corresponents, mitjançant l'anàlisi d'exemples prèviament resolts. Això li permet desenvolupar la capacitat de resoldre problemes nous aplicant la lògica i els patrons identificats a partir de dades reals. \\

\textbf{Avantatges i desavanantatge:}
Aquest mètode destaca per la seva alta precisió en problemes ben definits, ja que al treballar amb dades prèviament etiquetades pot assolir bons resultats en tasques de classificació i regressió. Una altra avantatge important és la facilitat per avaluar el rendiment dels models. A més, es tracta d'una àmplia àrea d'estudi amb una gran varietat d'algoritmes ben desenvolupats i optimitzats, com ara els arbres de decisió, els random forests, les màquines de vectors de suport (SVM) o les xarxes neuronals. Finalment, un cop entrenat adequadament, el model pot generalitzar el seu aprenentatge i fer prediccions útils sobre dades noves.\\

No obstant això, aquest enfocament també presenta alguns inconvenients significatius. El principal desavantatge és la seva forta dependència de conjunts de dades etiquetades, que sovint són costosos d'obtenir i preparar com las de medicines. Un altre problema freqüent és el sobreajustament (overfitting), que ocorre quan el model memoritza les dades d'entrenament en lloc d'aprendre patrons generals, la qual cosa provoca una perdua de raonament logica. Finalment, aquest mètode pot tenir dificultats per manejar certs tipus de dades no estructurades o problemes complexos, que podrien requerir quantitats molt grans de dades etiquetades per assolir un bon rendiment.


\subsubsection{Aprenentatge semi-supervisat}\label{subsubsec:Aprenentatge semi-supervisat}

L'aprenentatge semi-supervisat representa un punt intermig entre l'aprenentatge supervisat i el no supervisat, aprofitant tant dades etiquetades com no etiquetades per millorar l'eficiència dels models d'aprenentatge automàtic. Això funciona quan l'obtencio de les dades etiquetades son molt costoses i l'extraccio de les característiques son molt complexes.
\subsubsection{Aprenentatge no supervisat}\label{subsubsec:Aprenentatge no supervisat}

L'aprenentatge no supervisat és una branca de l'aprenentatge automàtic que s'utilitza quan no es disposa de dades etiquetades. A diferència de l'aprenentatge supervisat, on el model rep exemples amb les seves solucions correctes, en aquest cas l'algorisme ha de descobrir per si mateix l'estructura i els patrons, fent una diagnosticació agrupant les característiques similars que poden haber entre les dades.

Depenen dels tipus de  problemes, les dades s'organitzen de diferents maneres.
\begin{itemize}
 \item \textbf{Clustering:} Tècnica que agrupa les dades en funció de les seves similituds.
 \item \textbf{Anomaly detection:} Cerca patrons que no encaixen amb el comportament normal.
 \item \textbf{Association:} Cerca relacions i correlacions entre variables en grans conjunts de dades.
 \item \textbf{Autoencoders:} Els autoencoders són un tipus de xarxa neuronal artificial que aprèn a comprimir i reconstruir dades.
\end{itemize}


\subsubsection{Aprenentatge reforç}\label{subsubsec:Aprenentatge reforç}
L'aprenentatge per reforç és una branca de l'aprenentatge automàtic inspirada en la manera com els éssers vius aprenen mitjançant la interacció amb el seu entorn. Com per exemple quan començem a jugar un videojoc, en els jocs rebem senyals de reforços, de si completem un nivell ens ortoga un trofeu, de si matem certs enemics guanyem bonificacions. Amb aquest sistema de penalitzacio i recompenses guia al jugador a millorar les seves tecniques de videojocs , i això  el podem aplicar perfectament en la IA.
Totes les aprenantatges de reforç segueixen pragmaticament aquest esquema per l'aprenentage:
\begin{enumerate}
 \item \textbf{Acció}
 \item \textbf{Observació}
 \item \textbf{Recompensa}
 \item \textbf{Ajust d'estrategia}
\end{enumerate}


\subsubsection{Aprenentatge profund}
\begin{itemize}
    \item L'aprenentatge profund és una branca de l'aprenentatge automàtic que utilitza \nameref{sec:xarxa neuronal} amb múltiples capes per processar dades complexes i extreure'n característiques rellevants. Inspirat en el funcionament del cervell humà, aquest enfocament permet identificar patrons i analitzar dades de alta complexitat. Durant la fase d'identificació, empra un aprenentatge jeràrquic, és a dir, progressa gradualment des de característiques simples fins a patrons complexes.
\end{itemize}



\begin{figure}[h!]
    \centering
    \includegraphics[width=0.2\textwidth]{./figures/Aprenentatge.png}
    \caption{Les capes que te la IA per funcionar}
\end{figure}


\subsection{Potència computacional}\label{subsec:Potència computacional}
La intel·ligència artificial (IA) i la potència de càlcul estan profundament connectades. Sense maquinari potent, les aplicacions d’IA no podrien processar algoritmes complexos ni gestionar les enormes quantitats de dades que requereixen els models actuals. \\

Ultimament em vist el gran avanç que esta fent la IA, per tant la potencia computacional també creix consecutivament. D'aquesta manera la IA cada vegada es pareguera més a un huma i podrà ser més original alhora de crear contingut.\\

Una \hyperref[GPUs]{\textbf{GPUs}} té un papel molt important en la IA de tal manera que pot lograr a accelelar els calculs que fa. Per la cual cosa fa que el seu mercat puguess arribar fins a 53 milions de dolars en 2023 i s'estima a que arribara fins al 473 milions en 2033.\\

Tot això provoca un gran comerç i invertiment en la creació d'un chip exclusiu solament per la IA, donant llavor una rapideza encara més fascinant alhora de fer calculs.\\

Uns dels principals  protagonistes que treballen en el hardware de la IA, son mostrat per la següent imatge:
\begin{figure}[h!]
    \centering
    \includegraphics[width=0.5\textwidth]{./figures/Empreses.png}
    \caption{Les capes que te la IA per funcionar}
\end{figure}

\textbf{\Large GPUs}\label{GPUs}\\
Segons \href{https://support.microsoft.com/es-es/windows/todo-sobre-las-unidades-de-procesamiento-gr%C3%A1fico-gpu-e159bedb-80b7-4738-a0c1-76d2a05beab4}{Microsoft} una GPUs és:\\
  {\color{gray}``La unitat de processament de gràfics (GPU) d'un dispositiu Windows controla el treball relacionat amb els gràfics. Les GPU també es coneixen com a targetes de vídeo o targetes gràfiques.\\
  El treball relacionat amb els gràfics que manegen les GPU inclou els següents elements:
  \begin{itemize}
   \item El grafisme
   \item Efectes
   \item Video
   \item Videojocs
  \end{itemize}
  Hi ha dos tipus bàsics de GPU: GPU integrada i GPU discreta. Coneix els diferents tipus de GPU i troba el que s'ajusta a les teves necessitats:

  \textbf{GPU integrada}
  \begin{itemize}
   \item Les GPU integrades estan integrades en la placa base d'un dispositiu Windows.
   \item Les GPU integrades normalment no són tan potents com les GPU discretes, però són més eficients energèticament.
   \item Les GPU integrades sovint són menys costoses que les GPU discretes.
   \item Les GPU integrades permeten que els portàtils siguin més prims, lleugers i eficients.
   \item Les GPU Integrades són excel·lents per a alguns jocs, edició de vídeo lleuger o per treballar amb fotos.
  \end{itemize}

  \textbf{GPU discreta}
  \begin{itemize}
   \item Les GPU discretes són més grans que les GPU integrades i utilitzen més potència, però són les més adequades per a tasques de processament intensiu, com ara edició intensa de fotos i vídeos, treball de disseny i jocs.
   \item En els dispositius d'escriptori de Windows, les GPU discretes són una targeta independent que es connecta a la placa base del dispositiu Windows.
   \item Els ordinadors portàtils i tauletes de Windows també poden tenir GPU discrets directament a la placa base, però sovint s'afegeixen a la GPU incorporada. La GPU integrada s'utilitza per a tasques gràfiques més lleugeres, mentre que la GPU discreta s'utilitza per a tasques gràfiques més pesades. El canvi entre GPUs a partir de la tasca que es fa permet un equilibri entre rendiment i eficiència energètica.

   Els dispositius portàtils de Windows i les tauletes de Windows de vegades també poden tenir GPU discretament complementària a través de ports externs que poden suportar les velocitats necessàries per a una GPU. Normalment, la GPU es connecta a un port de Thunderbolt o més ràpid. ''

  \end{itemize}

}



\subsection{Software i Frameworks}\label{subsec:Software i Frameworks}

\subsubsection{Frameworks}
Segons \href{https://www.inesdi.com/blog/Frameworks-de-IA-que-debes-conocer/}{INESDI}:\\
{\color{gray}``Un Frameworks és, en el camp de la informàtica, una estructura conceptual que proporciona un conjunt d'eines, biblioteques i patrons de disseny per facilitar el desenvolupament de programari. En altres paraules, són marcs de treball que funcionen com un esquelet predefinit, i sobre els quals es pot construir una aplicació o programari.

Pel que fa als seus components, inclouen biblioteques de codi reutilitzables, mòduls predefinits, regles d'arxiu i directori, patrons de disseny i convencions de codificació.''}\\

Un Frameworks, en diferencia, de la biblioteca per el control total que pot tenir el usuari en tema estructura, i organitzacio dels codis, en canvi, una biblioteca està en mans del desenvolupador i només tens accés en els codis que et convenin de la biblioteca. Malgrat la utilitat que propociona no satisfeix la IA de la actualitat i han hagut de desarollar uns Frameworks especifics per la IA, de tal manera en que facilita el desarollament i l'entrenament dels models de la IA.
\subsubsection{Software}
El software és el conjunt de programes, instruccions i regles informàtiques que permeten executar tasques específiques en un ordinador o sistema.

La relació entre el software i la IA funciona de la següent manera: d'una banda, el software dóna utilitat pràctica a la IA en àmbits reals, permetent codificar les seves tècniques i processar les grans quantitats de dades que necessita. D'altra banda, la IA aporta automatització a les tasques repetitives del software, estalviant temps i recursos. Aquesta interacció crea un benefici mutu que forma un cicle perfecte, on cadascú millora i potencia l'altre.

\subsection{Optimització i Ajustos}\label{subsec:Optimització i Ajustos}
\begin{itemize}
 \item Ajust de hiperparàmetres: Per millorar la precisió del model.
 \item Validació i proves: Assegurar que la IA funcioni correctament en diferents escenaris.
\end{itemize}
\subsection{Aplicació pràctica}\label{subsec:Aplicació pràctica}
\begin{itemize}
 \item Integració amb sistemes existents: Per exemple, en robòtica, salut, finances, etc.
 \item Interfícies d'usuari: Com chatbots, aplicacions mòbils o sistemes de recomanació.
\end{itemize}
\subsection{Ètica i Regulació}\label{subsec:Ètica i Regulació}
\begin{itemize}
 \item Protecció de dades: Complir amb lleis com el GDPR.
 \item Biaixos i justícia: Assegurar que els models no discriminin.
\end{itemize}

Fonts:\href{https://blogs.uoc.edu/digitapia/the-european-unions-artificial-intelligence-act-explained/}{Bloc IA} \href{https://formaciooberta.eapc.gencat.cat/contingutsdelscursos/tdp/080_int_artificial/inici.html}{EAPC Wiki}
\href{https://www.ibm.com/think/topics/machine-learning}{IBM ML} \href{https://www.ultralytics.com/es/blog/understanding-the-impact-of-compute-power-on-ai-innovations}{La relació que estableix entre la IA i el maquinari }

\section{Que és una xarxa neuronal artificial/biologica?}\label{sec:xarxa neuronal}
Una xarxa neuronal artificial és un model computacional inspirat en el funcionament del cervell humà, utilitzat en el camp de la intel·ligència artificial (IA) i l'aprenentatge automàtic (machine learning). Està dissenyada per reconèixer patrons, prendre decisions i aprendre a partir de dades, sense ser programada explícitament per a cada tasca específica.
Si tenim una artificial també tindrem una biològica. Una xarxa neuronal biològica es refereix al sistema interconnectat de neurones (cèl·lules nervioses) en el cervell i el sistema nerviós dels éssers vius. Aquestes xarxes són la base de la cognició, l'aprenentatge i les funcions biològiques en humans i animals.
Aquí hi és una taula de comparació entre una biologica i artificial:

\begin{table}[h!]
\begin{tabular}{|l|l|l|}
\hline
\textbf{Aspecte} & \textbf{Xarxa neuronal biològica} & \textbf{Xarxa neuronal artificial} \\ \hline
\textbf{Base} & Cèl·lules vives (neurones). & Algoritmes matemàtics. \\ \hline
\textbf{Energia} & Baix consum (\textasciitilde20 watts). & Alt consum (GPUs/TPUs). \\ \hline
\textbf{Aprenentatge} & Plasticitat sinàptica. & Backpropagation + dades. \\ \hline
\textbf{Velocitat} & Lent (mil·lisegons). & Ràpid (nanosegons). \\ \hline
\end{tabular}
\caption{Comparativa entre xarxes neuronals biològiques i artificials}
\end{table}
Fons: \href{https://www.ibm.com/docs/es/spss-modeler/saas?topic=networks-basics-neural}{https://www.ibm.com/docs/es/spss-modeler/saas?topic=networks-basics-neural}
\section{Estructura d'una xarxa neuronal}\label{sec:3.6}
Una xarxa neurnal combina diverses capes de procesament i utilitza elements simples que operen en paral·lel, simulen i estan inspirades en els sistemes nerviosos biològics com hem explican en l'apartat \ref{sec:3.3}. Consta d'una capa d'entrada, seguit d'una o varies capes ocultes i finalment una capa de sortida. Les capes estan interconectades mitjançant nodes o neurones; cada capa utilitza la sortida de la capa anterior com a entrada.

\begin{itemize}
 \item \textbf{Capa d'entrada:}La capa d'entrada és la primera capa que reb directament la informació d'entrada que es prcessarà. Aquesta capa no realitzarà càlculs complexos, simplement transmet les dades a les capes seguents per fer el processament.
 \item \textbf{Capes ocultes:}Les capes ocultes són les capes que estan entre la capa d'entrada i la de sortida, aquestes capes contenen unitats no observables. La seva funció principal és processar les dades de la capa d'entrada per extraure característiques i patrons complexos. Aquestes capes són els que realitzan els càlculs i permet que la xarxa aprengui relacions no lineals i representacions abstracts de les dades, això és molt important per fer tasques complexes com el reconeixement de patrons.
 \item \textbf{Capa de sortida:}La capa de sortida és l'última capa que forma una xarxa neuronal i és l'encarregada de produïr la predicció final del model. Aquesta capa utilitza la informació que ha processat la o les capes ocultes i la transforma a travès d'una funció activa per generar una sortida, que pot ser una predicció numèrica, una clasificació o qualsevol altre resultat.
 \end{itemize}


\begin{figure}[h!]
    \centering
    \includegraphics[width=0.5\textwidth]{./figures/xarxa.png}
    \caption{Estructura d'una xarxa neuronal}
\end{figure}

Fons: \href{https://msmk.university/hidden-layer/}{https://msmk.university/hidden-layer/}

\href{https://www.linkedin.com/advice/0/what-some-examples-linear-nonlinear-models-real-world?lang=es&lang=es&originalSubdomain=es}{Article Linkedin}

\section{Funció d'activació}\label{sec:3.5.3}
Les funcions d'activació són un component integral de les xarxes neuronals que els permeten aprendre patrons complexos en les dades. Transformen el senyal d'entrada d'una neurona en un senyal de sortida que passa a la capa següent. Sense funcions d'activació, les xarxes neuronals es limitarien a modelar únicament relacions lineals entre entrades i sortides, és a dir, introdueixen la no linealitat i produeixen la sortida de la neurona.

\subsection{Funció sigmoide}\label{3.7.1}
Una funció d'activació molt coneguda és la funció sigmoide. La seva fórmula és:
\[ \sigma(x) = \frac{1}{1 + e^{-x}} \]

Aquesta funció matemàtica transforma qualsevol valor d'entrada real en un valor que està 0 i 1. La seva forma característica és una curva en forma de ``S''. Si el valor de $x$ que introduïm a la funció és molt gran o fins infinit $(\infty)$, llavors\ $\sigma$ serà 1; en canvi si és molt petit o menys infinit $(-\infty)$,\ $\sigma$ serà 0, i si x = 0,  $\sigma$  serà 0,5.

\begin{figure}[h!]
    \centering
    \includegraphics[width=0.5\textwidth]{./figures/grafica_sigmoide.png}
    \caption{Gràfica de la funció sigmoide}
\end{figure}

\textbf{Avantatges:}
La avantatge principal de la funció sigmoide és la seva suavitat i la facilitat de derivació. La funció és diferenciable en tots els punts, cosa que facilita el càlcul de gradients, això vol dir que permet als algoritmes d'optimització treballar amb la funció de manera eficient. A més, la funció és monòtona creixent, ho que significa que una entrada major sempre produirà una sortida major, fent que sigui útil per modelar relacions de causa i efecte.

\textbf{Desavantatges:}
Tanmateix, la funció sigmoide també té algunes desavantatges. Un dels seus problemes és que la funció es satura els gradients. Quan els valors d'entrades són grans, ho que significa que la derivada de la funció s'apropa a 0 i l'aprenentatge es relentitza. Un altre problema és que aquesta funció no és simètrica, causant a les entrades negatives i positives es processin de mandera diferent, això pot afectaral rendiment de la xarxa. Tot això dificulta el procès d'entrenament de la xarxa en la ràpida minimització de la funció d'error utilitzant l'algoritme de Gradient Descendent.
\subsection{Funció ReLU(Funció Uniat Rectificada Uniforme)}
La funció Unitat Rectificada Uniforme té la fòrmula seguent:
\[ f(x) = \max(0, x) \]

Aquesta funció té l'algoritme seguent: Si el valor d'entrada es menor que 0, mostra 0, si el valor d'entrada es major o igual que 0, mostrarà el valor d'entrada. Això vol dir que la funció és lineal si la entrada és més gran que 0 perquè la pendent és 1. Encara que la funció ReLU és lineal per a la mitat del seu espai d'entrada, tècnicament és una funció no lineal perquè té un punt no diferenciable en x = 0, on canvia bruscament respecte a x. Aquesta no linealitat permet a les xarxes neuronals apendre patrons complexos.

\begin{figure}[h!]
    \centering
    \includegraphics[width=0.5\textwidth]{./figures/ReLU.png}
    \caption{Gràfica de la funció ReLU}
\end{figure}

Tot i això, aquesta funció té una desavanantatge; si utilitzen la funció ReLU com a funció d'activació pasarà una cosa, i es que tots els valors negatius són 0, per tant en el procès de retropropagació, explicat a l'apartat \ref{sec:3.6.1}, no es produeix els valors d'ajust en les neurones negatives. Per solucionar aquest problema s'ha inventat una nova funció que es diu Leaky ReLU. Funciona igual com l funció ReLU, però té un valor determinat per les neurones negatives.

\begin{figure}[h!]
    \centering
    \includegraphics[width=0.5\textwidth]{./figures/leaky_ReLU.png}
    \caption{Gràfica de la funció Leaky ReLU}
\end{figure}

\subsection{Funció Softmax}
La funció Softmax, és una de les funcions que més s'utilitzen en xarxes neuronals i és especialment útil en el context dels problemes de clasificació multiclase. Aquesta funció opera sobre un vector que representa les previsions de cada clase, calculades per les capes anterior.

\begin{figure}[h!]
    \centering
    \includegraphics[width=0.5\textwidth]{./figures/Softmax.png}
    \caption{Gràfica de la funció Softmax}
\end{figure}

Per a un vector d'entrada x amb elements x1, x2,..., xC, la funció Softmax es defineix com:
\[f(x_i) = \frac{e^{x_i}}{\sum_{j=1}^{n} e^{x_j}}\]

El resultat de la funció Softmax és una distribució de probabilitat de la quàl la suma és 1. Cada element del resultat representa la probailitat de que l'entrada pertanyi a una clase determinada. L'ús d'aquesta funció garantitza de que tots els valors de la sortida siguin positius. Això és molt important perquè les porbabilitats no poden ser negatives.

\begin{figure}[H]
    \centering
    \includegraphics[width=0.5\textwidth]{./figures/representacio_Softmax.png}
    \caption{Representació de la funció Softmax}
\end{figure}

Fons: \href{https://msmk.university/hidden-layer/}{https://msmk.university/hidden-layer/}


\href{https://jacar.es/la-funcion-sigmoide-una-herramienta-clave-en-redes-neuronales/}{https://jacar.es/la-funcion-sigmoide-una-herramienta-clave-en-redes-neuronales/}

\section{Com funciona una xarxa neuronal?}\label{sec:3.8}
Ara que sabem quina estructura forma una xarxa neuronal artificial, toca entendre com funciona. Les neurones o nodes son els pilars més importants d'una xarxa neuronal. Cada neurona utilitza la entrada, la processa fent una suma ponderada entre els pesos i les entrades, després una funció d'activació, i pasa la sortida a altres neurones.

Les conexions(pesos i biaix) son la força de conexió entre dos neurones representades per un pes.
Els pesos: Son valors que determinen cuánta influència té la producció d'una neurona sobre un altre, marca la importància que té cada neurona.
Els biaixos: Són paràmetres adicionals que ajuda a ajustar els valors de les neurones. És un valor capaç d'apendre com el pes, això vol dir que el model pot utilitzar l'algoritme de retropropagació per a millorar els valors com per exemple els pesos i baixos.
El umbral: Si la sortida de qualsevol node individual és més gran que el valor del umbral, aquell node s'acivarà i envia dades a la seguent capa de la xarxa. En cas contrari, no es pasa ninguna dada a la seguent capa de la xarxa.

Hem de pensar en que cada node individual com el seu propi model de regressió lineal, compost per dades d'entrada, ponderacions, un umbral o un biaix, i una sortida. La fórmula per calcular els valors d'una xarxa neuronal és la seguent:

\[
\sum w_i x_i + \text{biaix} = w_1 x_1 + w_2 x_2 + w_3 x_3 + \text{biaix}
\]

$w_i$: Pesos
$x_i$: Entrades


Per entendre-ho millor, donarem un exemple: Imagina que vols anar a la platja a fer surf i la teva decisió dependrà d'aquestes 3 preguntes:

Hi ha oles?
Hi ha poca gent?
Hi ha tauros?

Analitzarem en detall com funciona un sol node de la xarxa neronal, utilitzant únicament valors binaris (0 y 1) com entrades, 0 voldrà dir ``No'' y 1 ``sí''.
LLavors suposent el següent cas:

$x_1$ = 1, les oles són bones
$x_2$ = 0, hi ha molta gent a la platja
$x_3$ = 1, Sí que tinc tempx

Ara, hem d'asignar algunes ponderacions per determinar la importància. Unes ponderacions majors significan que determinades variabbles son més importants per la decisió.

$w_1$ = 5, hi ha d'haveri moltes oles per poder surfejar
$w_2$ = 2, no t'importa que hi hagi molta gent
$w_3$ = 4, tens pors als taurons

Per últim, ambé suposem un valor umbral de 3, ho que es traduiria en un valor de biaix de -3. Amb totes aquestes entrades, ja podem substituir els valors en la nostre fòrmula per obtenir la sortida.



Les xarxes neuronals operen a travès d'un procès de dos pasos: La propagació directa i la retropropagació.

\subsection{Propagació cap a davant}
Durant la propagació cap a davan, les dades ingresen en la xarxa a travès de la capa d'entrada i flueixen secuencialment a travès de les capes ocultes fins a la capa de sortida. En cada neurona, els valors d'entrada del model es multiplican per els seus pesos corresponents i es sumen. Aquesta suma ponderada es pasa a travès d'una funció d'activació, explicada prèviament a l'apartat \ref{sec:3.5.3}. Aquest procès continua capa per capa, això acaba conduïnt cap a la predicció final en la capa de sortida.

Aquest procès és important per les seguents raons:
\begin{itemize}
 \item \textbf{Base per l'aprenentatge:} No es pot comprendre com aprenen les xarxes neuronals sense primer entendre com fan prediccions. La programació cap a davan és el requisit prèvi que s'ha de coneixer per comprendre la \ref{sec:3.6.1}{retroporgramació}, l'algoritme que permet l'aprenentatge.
 \item \textbf{Optimització:} En el cas de que una xarxa neuronal no funcioni bé, saber com flueixen les dades per la xarxa t'ajudarà a identificar i solucionar problemes.
 \item \textbf{Diseny del model:} Un diseny eficaç de la xarxa requereix comprendre com es distribueix la informació a travès de les configuracions de capes.
\end{itemize}

\subsection{Retropropagació}\label{sec:3.6.1}
Mentre que la programació directa fa prediccions, la retroporgramació és la forma en que la xarxa apren d'errors. Implica comparar la predicció de la xarxa amb el valor objectiu real i calcular un terme d'eror mitjançant una funció de pèrdua.
Aquest error es propaga enrere a travès de la xarxa, començant des de la capa de sortida. Durant aquest procès, la xarxa ajusta els pesos i els baixos de cada connexió en funció de la seva contribució a l'error, amb l'objectiu de minimitzar-lo.
Aquest procès iteratiu de càlcul d'erros i ajustament de pes permet a la xarxa d'aprenentatge profund millorar gradualment les seves prediccions.
\subsection{La regla de la cadena}
La regla de c



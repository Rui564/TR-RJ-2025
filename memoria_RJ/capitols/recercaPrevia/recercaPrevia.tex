\chapter{Recerca prèvia}
\label{c:intro}
\section{La Història de la Intel·ligència Artificial: Des dels Orígens fins Avui}
\subsection{El Naixement d'una Idea Revo lucionaria (1956)}
Tot va començar amb una pregunta provocadora d’Alan Turing, el geni matemàtic que va desxifrar Enigma, una màquina emprada pels nazis per codificar els seus missatges durant la Segona Guerra Mundial (1939-1945): "Podran les màquines pensar alguna vegada?". Aquesta qüestió va obrir les portes a un nou camp d’estudi. En 1956 John McCarthy, Marvin Minsky i d'altres especialistes van nominar oficialmenta el terme ``Inteligencia artificial'' durant la famosa conferencia de Darmouth, marcant el inici d'una nova era tecnologica.
\subsection{El Joc que ho va canviar tot: The Imitation Game/El test de Turing}
El nucli de la IA es basa en un experiment molt senzill però profund:El joc d'imitació(The imitation game), proposat per Alan Turing. Davant de la pregunta "Podran les màquines pensar alguna vegada?", Turing va dissenyar un joc que funcionava com a test per les màquines anomenat ``The Imitation Game''. Aquest test consistia en que un avaluador havia de començar una conversa en forma de textos escrits amb una persona i una máquina durant 5 minuts, aquest avaluador no sabia qui era qui i el seu objectiu era esbrinar qui era l'humà. Si la màquina aconseguia enganyar a l'avaluador pasaba el test i es reconeixia que la màquina havia aconseguit un nivell de comportament lingüístic equivalent a la d'un humà, i donava resposta a la pregunta d'Alan Turing. Al cap dels anys, el joc ha estat evolucionant i moderat per els prodigis de la humanitat fins que avui en dia es conegut com el test de Turing. Tot això va ser clau en  donar lloc al naixement de la IA i per l'avanç de la tecnologia.
\subsection{ELs grans fites de la IA}
\subsubsection{1997: La maquina que va vencer un campio}
 La supercomputadora Deep Blue desnvolupada per IBM va derrotar el campió mundial d’escacs, Garry Kasparov, demostrant que la IA podia superar els humans en jocs d’estratègia complexos.
\subsubsection{2022: L'explosió de la IA}
Milions d’usuaris van descobrir models com ChatGPT que podien escriure, traduir i programar amb un llenguatge gairebé humà, obrint nous horitzons en la interacció home-màquina.
\subsubsection{2025: La IA en tots els ambits}
    2025: La IA en Tots els Àmbits
    Avui, la IA està present en dibuix, contingut audiovisual, cotxes autònoms, medicina i molt més, amb models cada vegada més especialitzats i avançats.



(Fonts: \cite{McCarthy_Minsky_Rochester_Shannon_2006}, \cite{deep-blue},
\href{https://openai.com/index/chatgpt/}{ChatGPT (2022)},
\cite{10.1093/mind/LIX.236.433}
)

\section{Que és i com funciona la Ia?}
\subsection{Que és una IA?}
Hem estat parlant molt durant aquest treball sobre la IA, i ara que em après quina és la seva història, hem de saber que és. Doncs bé, podem definir la IA com sistemes de software i de hardware disenyats per humans que actúen en un la dimensió física o digital, es a dir, raonar sobre el coneixemen, processant la informació derivada de dades i prendre les millors decisions per assolir l'objectiu donat. O dit d'un altre manera, és un camp de la informàtica que consisteix en un conjunt de capacitats interectuals i cognitives expresades per un sistema informàtic creat pels humans que té com a proposit imitar la intel·ligència humana, com escriure poemes, reconeixer imatges, fer prediccions basades en dades i més funcions. Un exemple de IA i que tot el món coneix i utilitza és el ChatGPT, un chatbot impulsada per un model d'intel·ligència artificial generativa de la empresa OpenAI. Utilitza técniques de processament de llenguatge natural per comprendre preguntes fetes per l'usuari i generar respostes coherents en converses, simulant una interacció similar a la d'un humà. Aquesta IA ha avançat molt desde el seu llençament, ara pot generar o editar imatges, procesar audios, llegir arxius, comrendre imatges i molt més.

\subsection{Com funciona la IA?}
Una vegada que ja sabem que és una IA, ens toca entendre com funciona. Les intel·ligències arficials utilitzen algoritmes i models matemàtics per processar grans quantitats de dades i prendre accions basades en patrons i regles establertes a travès de l'aprenentatge automàtic o l'aprenentatge profund.
\begin{itemize}
 \item L'aprenentage automàtc consisteix en que els sistemes de IA analitzen grans quantitats de dades per identificar patrons i relacions, aprenent a realitzar tasques específics amb major precisió quan més informació reben.
 \item L'aprenentatge profund és una branca de l'aprenentatge profund, utilitza xarxes neuronals explicat en l'apartat \ref{sec:3.3} amb múltiples capes per procesar dades complexes i extraure característiques rellevants,això permet a la IA realitzar tasques com el reconeixement d'imatges.
\end{itemize}

\section{Que és una xarxa neuronal artificial/biologica?}\label{sec:3.3}
Una xarxa neuronal artificial és un model computacional inspirat en el funcionament del cervell humà, utilitzat en el camp de la intel·ligència artificial (IA) i l'aprenentatge automàtic (machine learning). Està dissenyada per reconèixer patrons, prendre decisions i aprendre a partir de dades, sense ser programada explícitament per a cada tasca específica.
Si tenim una artificial també tindrem una biologica.Una xarxa neuronal biològica es refereix al sistema interconnectat de neurones (cèl·lules nervioses) en el cervell i el sistema nerviós dels éssers vius. Aquestes xarxes són la base de la cognició, l'aprenentatge i les funcions biològiques en humans i animals.
Aquí hi és una taula de comparació entre una biologica i artificial:

\begin{table}[t!]
\begin{tabular}{|l|l|l|}
\hline
\textbf{Aspecte} & \textbf{Xarxa neuronal biològica} & \textbf{Xarxa neuronal artificial} \\ \hline
\textbf{Base} & Cèl·lules vives (neurones). & Algoritmes matemàtics. \\ \hline
\textbf{Energia} & Baix consum (\textasciitilde20 watts). & Alt consum (GPUs/TPUs). \\ \hline
\textbf{Aprenentatge} & Plasticitat sinàptica. & Backpropagation + dades. \\ \hline
\textbf{Velocitat} & Lent (mil·lisegons). & Ràpid (nanosegons). \\ \hline
\end{tabular}
\caption{Comparativa entre xarxes neuronals biològiques i artificials}
\end{table}




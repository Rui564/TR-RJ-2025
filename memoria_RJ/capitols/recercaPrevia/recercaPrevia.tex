\chapter{Recerca prèvia}
\label{c:intro}
\section{La Història de la Intel·ligència Artificial: Des dels Orígens fins Avui}
\subsection{El Naixement d'una Idea Revolucionaria (1956)}
Tot va començar amb una pregunta provocadora d’Alan Turing, el geni matemàtic que va desxifrar Enigma, una màquina emprada pels nazis per codificar els seus missatges durant la Segona Guerra Mundial (1939-1945): "Podran les màquines pensar alguna vegada?". Aquesta qüestió va obrir les portes a un nou camp d’estudi. En 1956 John McCarthy, Marvin Minsky i d'altres especialistes van nominar oficialmenta el terme ``Inteligencia artificial'' durant la famosa conferencia de Darmouth, marcant el inici d'una nova era tecnologica.
\subsection{El test de Turing}
Després de la pregunta que es va fer Alan Turing sobre si les màquines podien pensar, va elaborar una prova per les màquines anomenat ``Test de turing''.
El test de Turing s'utilitzaba com un estàndar per mesurar la intel·ligència artifficial i la capacitat d'una màquina per realitzar tasques que requerien intel·ligència humana.
Aquest test consistia en que un avaluador havia de començar una conversa en forma de textos escrits amb una persona i una máquina durant 5 minuts, aquest avaluador no sabia qui era qui i el seu objectiu era esbrinar qui era l'humà. Si la màquina aconseguia engañar a l'avaluador pasaba el test i es reconeixia que la màquina havia aconseguit un nivell de comportament llingüístic igual a la d'un humà, arribant a la conclusió que una màquina sí podien pensar.
El test de turing segueix sent un objectiu molt important per la investigació de les intel·ligències artificials i ha sigut el punt de partida per a moltes de les tecnologies de processament de llengüatge natural.

\subsection{ELs grans fites de la IA}
\subsubsection{1997: La maquina que va vencer un campio}
 La supercomputadora Deep Blue desnvolupada per IBM va derrotar el campió mundial d’escacs, Garry Kasparov, demostrant que la IA podia superar els humans en jocs d’estratègia complexos.
\subsubsection{2022: L'explosió de la IA}
Milions d’usuaris van descobrir models com ChatGPT que podien escriure, traduir i programar amb un llenguatge gairebé humà, obrint nous horitzons en la interacció home-màquina.
\subsubsection{2025: La IA en tots els ambits}
    2025: La IA en Tots els Àmbits
    Avui, la IA està present en dibuix, contingut audiovisual, cotxes autònoms, medicina i molt més, amb models cada vegada més especialitzats i avançats.
\subsection{El Joc que ho va canviar tot: The Imitation Game}
El nucli de la IA es basa en un experiment molt senzill però profund:El joc d'imitació(The imitation game),proposat per Turing.En aquest joc, un jutge ha d’identificar quin dels dos ``humans'' que ha estat conversant es una maquina  i quin es huma.Si la IA pot enganyar el jutge, aleshores es pot dir que "pensa". Aquest concepte va ser fonamental per demostrar que les màquines podrien arribar a raonar com els humans.



(Fonts: \href{https://raysolomonoff.com/dartmouth/boxa/dart564props.pdf}{Dartmouth Proposal (1956)}, \href{https://www.ibm.com/history/deep-blue}{Deep Blue (1997)}, \href{https://openai.com/index/chatgpt/}{ChatGPT (2022)}, \href{https://academic.oup.com/mind/article-abstract/LIX/236/433/986238?redirectedF}{Turing's Imitation Game (1950)})

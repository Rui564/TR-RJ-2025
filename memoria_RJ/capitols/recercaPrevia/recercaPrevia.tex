\chapter{Recerca prèvia}
\label{c:intro}
\section{La Història de la Intel·ligència Artificial: Des dels Orígens fins Avui}
\subsection{El Naixement d'una Idea Revo lucionaria (1956)}
Tot va començar amb una pregunta provocadora d’Alan Turing, el geni matemàtic que va desxifrar Enigma, una màquina emprada pels nazis per codificar els seus missatges durant la Segona Guerra Mundial (1939-1945): "Podran les màquines pensar alguna vegada?". Aquesta qüestió va obrir les portes a un nou camp d’estudi. En 1956 John McCarthy, Marvin Minsky i d'altres especialistes van nominar oficialmenta el terme ``Inteligencia artificial'' durant la famosa conferencia de Darmouth, marcant el inici d'una nova era tecnologica.
\subsection{El Joc que ho va canviar tot: The Imitation Game/El test de Turing}
El nucli de la IA es basa en un experiment molt senzill però profund:El joc d'imitació(The imitation game), proposat per Alan Turing. Davant de la pregunta "Podran les màquines pensar alguna vegada?", Turing va dissenyar un joc que funcionava com a test per les màquines anomenat ``The Imitation Game''. Aquest test consistia en que un avaluador havia de començar una conversa en forma de textos escrits amb una persona i una máquina durant 5 minuts, aquest avaluador no sabia qui era qui i el seu objectiu era esbrinar qui era l'humà. Si la màquina aconseguia enganyar a l'avaluador pasaba el test i es reconeixia que la màquina havia aconseguit un nivell de comportament lingüístic equivalent a la d'un humà, i donava resposta a la pregunta d'Alan Turing. Al cap dels anys, el joc ha estat evolucionant i moderat per els prodigis de la humanitat fins que avui en dia es conegut com el test de Turing. Tot això va ser clau en  donar lloc al naixement de la IA i per l'avanç de la tecnologia.
\subsection{ELs grans fites de la IA}
\subsubsection{1997: La maquina que va vencer un campio}
 La supercomputadora Deep Blue desnvolupada per IBM va derrotar el campió mundial d’escacs, Garry Kasparov, demostrant que la IA podia superar els humans en jocs d’estratègia complexos.
\subsubsection{2022: L'explosió de la IA}
Milions d’usuaris van descobrir models com ChatGPT que podien escriure, traduir i programar amb un llenguatge gairebé humà, obrint nous horitzons en la interacció home-màquina.
\subsubsection{2025: La IA en tots els ambits}
    2025: La IA en Tots els Àmbits
    Avui, la IA està present en dibuix, contingut audiovisual, cotxes autònoms, medicina i molt més, amb models cada vegada més especialitzats i avançats.



(Fonts: \href{https://raysolomonoff.com/dartmouth/boxa/dart564props.pdf}{Dartmouth Proposal (1956)}, \href{https://www.ibm.com/history/deep-blue}{Deep Blue (1997)}, \href{https://openai.com/index/chatgpt/}{ChatGPT (2022)}, \href{https://academic.oup.com/mind/article-abstract/LIX/236/433/986238?redirectedF}{Turing's Imitation Game (1950)})


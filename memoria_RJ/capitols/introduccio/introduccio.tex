\chapter{Introducció}
\label{c:intro}
En el nostre dia i dia consumim continguts digitals de manera constant: visualitzem vídeos curts, fem fotografies, realitzem revisions corporals, juguem a videojocs o naveguem per Internet. Darrere de moltes d’aquestes activitats hi trobem la presència de la intel·ligència artificial (IA), que juga un paper rellevant en àmbits tan diversos com els suggeriments de continguts audiovisuals, el diagnòstic mèdic, la gestió i distribució de la informació o l’optimització d’imatges.

Aquest protagonisme fa que, actualment, la IA sigui cada vegada més demandada en els entorns industrials i professionals. Tanmateix, la probesa de coneixements especialitzats i la complexitat pròpia d’aquest camp fan que relativament poques persones decideixin dedicar-s’hi.

Amb aquesta situació com a punt de partida, la nostra intenció és introduir-nos en el camp de la intel·ligència artificial, aprofitant l’oportunitat que ens ofereix el treball de recerca i amb la voluntat de treure’n el màxim profit. La nostra motivació principal és adquirir una formació que ens permeti preparar-nos als reptes del futur i obtenir un avantatge competitiu en la nostra trajectòria acadèmica i professional.






\section{Motivacions}
Un cop explicat l’objectiu del treball, volem aclarir les motivacions i les raons que ens empenyen a dur-lo a terme:

\begin{itemize}
 \item La principal raó que ens va portar a aquest tema és la nostra afició per la informàtica. En un futur ens agradaria aprofundir-hi i continuar treballant-hi en un grau, màster o doctorat. Per això, aprofitem l’oportunitat del Treball de Recerca per començar a preparar-nos.

 \item Una altra raó que ens va permetre escollir aquest tema és la gran quantitat i diversitat de recursos disponibles a Internet. A més, el nostre tutor, Fernando García, és matemàtic i informàtic, fet que ens facilita la feina. També un amic nostre ens va oferir recursos externs per elaborar la part pràctica de la xarxa neuronal amb el full de càlcul.

 \item El nostre tutor ens va proposar diversos temes, i finalment vam escollir aquest per l’interès que ens va despertar.

 \item El nostre interès per la programació ha estat també una motivació fonamental en aquest treball.

 \item La voluntat de construir una eina amb una mirada cap al futur, aplicant els coneixements adquirits i amb la determinació de fer un treball rigorós i ambiciosament plantejat.
\end{itemize}

\section{Estructura de la memòria}
El primer que ens vam proposar va ser establir uns objectius clars per orientar-nos en el desenvolupament del treball. Aquests es presenten en el capítol 2, \nameref{c:objectius}.

Un cop definits els objectius, calia adquirir coneixements previs i actualitzats per comprendre el funcionament de les xarxes neuronals. Aquesta recerca es presenta en el capítol 3, \nameref{c:recerca prèvia}.

Després, vam iniciar una nova recerca per decidir quines eines utilitzar, amb l’objectiu de guanyar temps i millorar la qualitat del treball. Tot i que no sigui d’una complexitat excessiva, és un treball que requereix esforç i dedicació. Aquesta part es mostra en el capítol 4, \nameref{c:Metodologia}.

Tot i el treball constant i l’esforç que hem invertit, l’experiència ha estat majoritàriament positiva. Gràcies a aquest procés hem pogut enriquir-nos amb coneixements que no s’adquireixen a classe i guanyarem un avantatge en iniciar els estudis universitaris. A més, aquest treball ens ajudarà a millorar la redacció i el nivell de la llengua, que és un dels nostres punts febles.

Els assoliments i resultats obtinguts es recullen en el capítol 5, \nameref{c:Resultats}.

Finalment, en el capítol 6, \nameref{c:conclusions}, exposem les conclusions a què hem arribat i els possibles plans de futur relacionats amb el tema.




\chapter{Introducció}
\label{c:intro}
Quan vam iniciar a triar sobre el tema del Treball de recerca teniem clar que voliem fer-ho sobre alguna branca de la informàtica, però no sabiem molt clar que fer exactament.

Finalment, el nostre tutor del Treeball de recerca ens va donar un molt bon tema per treballar i que ens va cridar l'atenció als dos, que és l'estudi de les xarxes neuronals artificials que té una intel·ligència artificial.

La intel·ligència artificial és una nova tecnologia que ha anat evolucionant amb molta rapidesa durant l'últim segle. És una màquina capaç de simular el comportament humà i realitzar tasques com el raonament, l'aprenentatge, el reconeixement d'una imatge o veu, etc... Actualment, les intel·ligències artificials poden apendre per ells mateixos a partir de dades, algoritmes i xarxes nuronals. Aquest darrer serà el component que més aprofunditzarem en aquest treball, ja que és ho més important per complir el nostre objectiu d'aquest treball.

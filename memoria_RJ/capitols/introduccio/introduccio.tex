\chapter{Introducció}
\label{c:intro}
1956; ``John McCarthy, Marvin Minsky i altres llegendes defineixen el terme "intel·ligència artificial" en un estiu que marcaria el futur tecnològic''.{https://raysolomonoff.com/dartmouth/boxa/dart564props.pdf}1997;``Per primera vegada, una màquina venç a un campió del món en un joc d'estratègia, sorprenent el món''.{https://www.ibm.com/history/deep-blue}2022;``Milions d'usuaris descobreixen com la IA pot escriure, traduir o programar amb un llenguatge gairebé humà.''.{https://openai.com/index/chatgpt/} 2025;  Avui en dia, diversos models d'IA apegaren tenint diverses funcionalitats en camps específiques: ``Dibuix, contingut audiovisual, cotxes autònoms i molt més''.

Tot això va sorgir una pregunta revolucionària que se li va acudir a Alan Turing, el geni matemàtic que va desxifrar la màquina Enigma (un dispositiu de xifratge utilitzat pels nazis per codificar missatges) en un món que encara estba en procés de recuperació després de la Segona Guerra Mundial (1914-1954); ``Podran les màquines pensar alguna vegada?''.I l'eix del destí de la indústria IA va arrancar mitjançant un joc ximple, però sòlida en què demostra que una IA també pot pensar i raonar com un humà. El joc ``Imitation game'' en què consistia que hi havia 3 papers: Un home, una dona i un jutge. Un dels dos humans seria interpretada per la IA i el jutge hauria d'identificar quins dels tres és la IA.{https://academic.oup.com/mind/article-abstract/LIX/236/433/986238?redirectedF}

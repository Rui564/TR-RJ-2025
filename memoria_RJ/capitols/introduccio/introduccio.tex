\chapter{Introducció}
\label{c:intro}
<<<<<<< HEAD
1956; ``John McCarthy, Marvin Minsky i altres llegendes defineixen el terme "intel·ligència artificial" en un estiu que marcaria el futur tecnològic''.{https://raysolomonoff.com/dartmouth/boxa/dart564props.pdf}1997;``Per primera vegada, una màquina venç a un campió del món en un joc d'estratègia, sorprenent el món''.{https://www.ibm.com/history/deep-blue}2022;``Milions d'usuaris descobreixen com la IA pot escriure, traduir o programar amb un llenguatge gairebé humà.''.{https://openai.com/index/chatgpt/} 2025;  Avui en dia, diversos models d'IA apegaren tenint diverses funcionalitats en camps específiques: ``Dibuix, contingut audiovisual, cotxes autònoms i molt més''.

Tot això va sorgir una pregunta revolucionària que se li va acudir a Alan Turing, el geni matemàtic que va desxifrar la màquina Enigma (un dispositiu de xifratge utilitzat pels nazis per codificar missatges) en un món que encara estba en procés de recuperació després de la Segona Guerra Mundial (1914-1954); ``Podran les màquines pensar alguna vegada?''.I l'eix del destí de la indústria IA va arrancar mitjançant un joc ximple, però sòlida en què demostra que una IA també pot pensar i raonar com un humà. El joc ``Imitation game'' en què consistia que hi havia 3 papers: Un home, una dona i un jutge. Un dels dos humans seria interpretada per la IA i el jutge hauria d'identificar quins dels tres és la IA.{https://academic.oup.com/mind/article-abstract/LIX/236/433/986238?redirectedF}
=======
Les xarxes neuronals artificials son uns dels fenòmens tecnológics més revolucionaris en la Intel·ligència artificial i estan inspirades en el funcionament de les neurones del cervell humà. Aquesta estructura artifical que mostra un gran aprenentatge automàtic i la resolució de problemes complexos. Tot i així, encara hi ha molts reptes per sobrepassar i aixó constarà molt de treball, especialment en l'àmbit del raonament humà i l'adaptabilitat d'aquest.\\
Aquest treball té com a objectiu aprofundir en l'estudi de les xarxa neuronal amb un enfocament especial en `` model de xarxa neuronal''. Volem analitzar la part teòrica i pràctica amb la finalitat de explorar la funcionalitat d'aquesta tecnologia i possibles millores.\\
Per aconseguir aquest objectiu,hem estructurat tres blocs de recerca:
\section{Blocs de recerca}
\subsection{Part teorica}
Fer una recerca dels potents eines que podem fer ús tant a nivell matemàtic com a nivell  d'infromàtica.
\subsection{Aplicacions practiques}
Fer una recerca dels models de xarxes neuronals i posar-los en pràctica per experimentar diferents funcionalitats que pot haver-hi.
\subsection{Part practica }
Construir una xarxa neurnal pel nostre compte amb diferents eines que disposem.

Vam escollir aquest ambit per diversos raons pero uns dels principals raons es que volem unir-nos en aquest viatge cap  a infinites possibilitat, i per l'altra banda tenim al nostre tutor que té una gran experiència en aquest tema i ens donaria un gran suport per avançar en aquesta recerca.\\
En aquesta recerca esperem trobar-nos reptes amb molta dificultat, per superar els nostres límits i poder crear un jo millor i desde el punt de partida ja poder tenir un gran advantatge respecte d'altres alumnes d'informatica.

















>>>>>>> refs/remotes/origin/main

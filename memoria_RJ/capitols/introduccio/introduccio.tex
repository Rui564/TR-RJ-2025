\chapter{Introducció}
\label{c:intro}

Amb el pas del temps, la visio futurista del gran escritor \textit{Fredric Brown} en el relat \emph{Answer} cada vegada resulta menys ficticia. És probable de que algun dia nosaltres els humans puguem arribar a tal nivell. Malgrat els avenços que estem aconseguint en l'ambit IA encara ens falta un llarg cami per recorrer, que ens separa d'aquell somni esperada.

Actualment la IA ja no és un concepte ciencia-ficcio, sino ja hi es en el nostre dia quotidiana, des d’assistents virtuals fins a sistemes de diagnòstic mèdic avançat.

Tanmateix, les inteligencia artificials com chatgpt i deepseek a provocat problemes etics i morals. Han sorgit molts problemes estil: Qui controla aquesta tecnologia? Quins treballs seran substituït? Com afronten els riscos de desevolupament descontrolat?

En aquest context voldriem construir una xarxa neuronal, no una IA per la seva complexibilitat i dificultat que esta fora del nostre nivell.

Per dur a terme aquest estudi, vam seleccionar tres models de xarxes neuronals diferents: una implementada en Python, una altra basada en fulls de càlcul i una tercera aplicada a un exemple real del joc Mobile Legends: Bang Bang.

\section{Motivacions:}
Un cop explicat el que voliem fer, voldriem aclarir les Motivacions i els raons que ens empuja cap a endavant en l'elaboració d'aquest treball:

\begin{itemize}
 \item La principal raó que ens va portar a aquest tema es per la nostra afició cap la informatica, i en un futur voldriem a profundir el tema i continuar treballant-hi en un grau o master o doctorat per tant ens agradaria anar-hi ja preparat aprofitant el temps per fer el TR.

 \item Una altre gran rao que ens van permetre fer el treball es gracies a la gran quantitat i diversitat de  recursos que ens oferia a internet, també el nostre tutor Fernando Garcia era un matematic e informatic per tant ens facilitaba la feina, a la vegada un amic nostre ens va oferir recursos externes per elaborar la part practica xarxa neuronal del full de calculs.

 \item El nostre tutor ens va plantejar diversos temes per treballar-hi i finalment ens vam quedar amb aquest tema tant interessant.

 \item La voluntat de voler construir una eina futuristic amb els coneixements que estem adquirint i disposat a fer un gran treball per afrontar-lo.

\end{itemize}


\section{Estructura de la memoria}

El primer pas que vam donar va ser donar-nos uns Goals (Objectius) per fixar i orientar-nos en el nostre treball, presentada en el capitol 2\nameref{c:objectius}.

Després d'haver fixat el nostre treball, per tant, saber que fer, ens toca adquirir coneixements previs i coneixements actualitzats per assabentar-nos com funciona les xarxes neuronals i poder donar lloc les creacions de les xarxes neuronals. Tot aquesta recerca s'explica en el capitol 3 \nameref{c:recerca prèvia}

Una vegada que hem finalitzat amb la recerca prèvia, ens caldra iniciar un altre recerca per saber quins tipus d'eines farem utilitzar per guanyar temps i aumentar la calidad del treball, ja que encara que el treball no sigui de complexibilitat excessiva, és un treball elaborada i exigeix un gran esforç. Tota aquesta recerca es mostra en el capitol 4 \nameref{c:Metodologia}.

Tot i que el treball constant que hem de invertir i el gran esforç que farem el treball és mayoritariament positiva per part nostra. Graciés  aquest treball ens podrem enrequir-nos de coneixements que no ens donara a classe i podrem treure una gran advantatge al començar un grau, es mes ens ajudara en millorar la nostra redactacio i pujar el nivell de la llengua que es el nostre punt feble.

Dels assoliments que aconseguim, ilustrarem un apartat on registra tots els resultats que aconseguirem, explicada en el capitol 5 \nameref{c:Resultats}

Finalment per finalitzar el treball en el capitol 6 \nameref{c:conclusions} hem donat les conclusions que hem arribat a elaborar i els futurs plans que podem tenir.



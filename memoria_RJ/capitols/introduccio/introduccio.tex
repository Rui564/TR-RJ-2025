\chapter{Introducció}
\label{c:intro}

Amb el pas del temps, la visió futurista del gran escriptor \textit{Fredric Brown} en el relat \emph{Answer} cada vegada resulta menys fictícia. És probable que algun dia els humans puguem arribar a assolir aquell nivell. Malgrat els avenços que estem aconseguint en l'àmbit de la intel·ligència artificial (IA), encara ens queda un llarg camí per recórrer abans d’arribar a aquell somni tan esperat.

Actualment, la IA ja no és un concepte de ciència-ficció, sinó una realitat present en el nostre dia a dia: des d’assistents virtuals fins a sistemes de diagnòstic mèdic avançat.

Tanmateix, les intel·ligències artificials com ChatGPT o DeepSeek han generat qüestions ètiques i morals importants. S’han plantejat problemes com ara: qui controla aquesta tecnologia?, quins llocs de treball seran substituïts?, com es poden afrontar els riscos d’un desenvolupament descontrolat?

En aquest context, nosaltres volem construir una xarxa neuronal senzilla. No pretenem crear una IA completa, ja que la seva complexitat està fora del nostre  nivel actual.

\section{Motivacions}
Un cop explicat l’objectiu del treball, volem aclarir les motivacions i les raons que ens empenyen a dur-lo a terme:

\begin{itemize}
 \item La principal raó que ens va portar a aquest tema és la nostra afició per la informàtica. En un futur ens agradaria aprofundir-hi i continuar treballant-hi en un grau, màster o doctorat. Per això, aprofitem l’oportunitat del Treball de Recerca per començar a preparar-nos.

 \item Una altra raó que ens va permetre escollir aquest tema és la gran quantitat i diversitat de recursos disponibles a Internet. A més, el nostre tutor, Fernando García, és matemàtic i informàtic, fet que ens facilita la feina. També un amic nostre ens va oferir recursos externs per elaborar la part pràctica de la xarxa neuronal amb el full de càlcul.

 \item El nostre tutor ens va proposar diversos temes, i finalment vam escollir aquest per l’interès que ens va despertar.

 \item El nostre interès per la programació ha estat també una motivació fonamental en aquest treball.

 \item La voluntat de construir una eina amb una mirada cap al futur, aplicant els coneixements adquirits i amb la determinació de fer un treball rigorós i ambiciosament plantejat.
\end{itemize}

\section{Estructura de la memòria}
El primer que ens vam proposar va ser establir uns objectius clars per orientar-nos en el desenvolupament del treball. Aquests es presenten en el capítol 2, \nameref{c:objectius}.

Un cop definits els objectius, calia adquirir coneixements previs i actualitzats per comprendre el funcionament de les xarxes neuronals. Aquesta recerca es presenta en el capítol 3, \nameref{c:recerca prèvia}.

Després, vam iniciar una nova recerca per decidir quines eines utilitzar, amb l’objectiu de guanyar temps i millorar la qualitat del treball. Tot i que no sigui d’una complexitat excessiva, és un treball que requereix esforç i dedicació. Aquesta part es mostra en el capítol 4, \nameref{c:Metodologia}.

Tot i el treball constant i l’esforç que hem invertit, l’experiència ha estat majoritàriament positiva. Gràcies a aquest procés hem pogut enriquir-nos amb coneixements que no s’adquireixen a classe i guanyarem un avantatge en iniciar els estudis universitaris. A més, aquest treball ens ajudarà a millorar la redacció i el nivell de la llengua, que és un dels nostres punts febles.

Els assoliments i resultats obtinguts es recullen en el capítol 5, \nameref{c:Resultats}.

Finalment, en el capítol 6, \nameref{c:conclusions}, exposem les conclusions a què hem arribat i els possibles plans de futur relacionats amb el tema.




\chapter{Introducció}
\label{c:intro}
Al nostre dia a dia consumim continguts digitals constantment: veiem vídeos curts, fem fotos, fem revisions corporals, juguem videojocs, naveguem per Internet, entre moltes altres activitats. Darrere de tot això hi ha la Intel·ligència Artificial (IA), que juga un paper important en àmbits tan diversos com els continguts audiovisuals, el diagnòstic mèdic, la gestió i distribució de la informació o l’optimització de la qualitat d’imatges.

Aquest protagonisme fa que la IA sigui cada vegada més demandada en entorns industrials i professionals. Tot i això, la manca de formació específica i la complexitat del camp dificulten el seu accés.

Partint d’aquesta situació, volem endinsar-nos en el món de la intel·ligència artificial, aprofitant l’oportunitat que ens ofereix el treball de recerca i amb la voluntat de treure’n el màxim profit. La nostra motivació principal és adquirir coneixements que ens ajudin a preparar-nos per als reptes del futur i que siguin útils tant acadèmicament com professionalment.

En aquest treball ens centrarem en les xarxes neuronals, que són essencials en la intel·ligència artificial, ja que permeten processar grans quantitats de dades i generar resultats com prediccions i identificació de patrons. Analitzarem com funcionen els algoritmes d’entrenament i algunes aplicacions pràctiques amb la finalitat de demostrar si el full de càlculs té la capacitat de competir amb una xarxa neuronal construida amb un llenguatge de programació en presició.
%%%%%%%%%%%%%%%%%%%%%%%%%%%%%%%%%%%%%%%%%%%%%%%%%%%%%%%%%%%%%%%%%%%%%%%
%%%%%%%%%%%%%%%%%%%%%%%%%%%%%%%%%%%%%%%%%%%%%%%%%%%%%%%%%%%%%%%%%%%%%%%

%%%%%%%%%%%%%%%%%%%%%%%%%%%%%%%%%%%%%%%%%%%%%%%%%%%%%%%%%%%%%%%%%%%%%%%
%%%%%%%%%%%%%%%%%%%%%%%%%%%%%%%%%%%%%%%%%%%%%%%%%%%%%%%%%%%%%%%%%%%%%%%


\section{Motivació}
Ara que ja hem presentat la temàtica del nostre Treball de Recerca (TR), expliquem quines són les motivacions que ens empenyen a dur-lo a terme:

\begin{itemize}
\item La principal raó que ens va portar a aquest tema és la nostra afició per la informàtica. En un futur ens agradaria aprofundir-hi i continuar treballant-hi en un grau, màster o doctorat. Per això aprofitem l’oportunitat del TR per començar a preparar-nos.

\item Una altra raó que ens va animar a triar aquest tema són els recursos que tenim a la nostra disposició. D'una banda, el nostre tutor té contactes amb professors de la Facultat de Matemàtiques i de la Facultat d’Informàtica de la Universitat de Barcelona i de la Universitat Rovira i Virgili. A més, un amic nostre ens va oferir recursos externs per elaborar la part pràctica de la xarxa neuronal amb el full de càlcul.

\item Conjuntament amb el tutor vam decidir que el TR havia de ser una oportunitat per aprendre a fer servir recursos útils per als nostres estudis universitaris: un entorn de treball col$\cdot$laboratiu, dominar llenguatges de programació i utilitzar un editor de text apropiat per a textos científics.

\item La voluntat de construir una eina amb mirada cap al futur, aplicant els coneixements adquirits i amb la determinació de fer un treball rigorós i ambiciós.
\end{itemize}

\section{Estructura de la memòria}

Després d'aquesta breu introducció, al capítol~\ref{c:objectius}:~\nameref{c:objectius}. es presenten els objectius del nostre TR. A continació, al capítol~\ref{c:recerca prèvia}:~\nameref{c:recerca prèvia}, es presenten els coneixements previs necessaris per a comprendre el funcionament de les xarxes neuronals.
Després, vam iniciar una nova recerca per decidir quines eines utilitzar, amb l’objectiu de guanyar temps i millorar la qualitat del treball. Tot i que no sigui d’una complexitat excessiva, és un treball que requereix esforç i dedicació. Aquesta part es mostra en el capítol~\ref{c:Metodologia}:~\nameref{c:Metodologia}. Un cop hem presentat com treballarem, toca presentar els resultats obtinguts al ~\ref{c:Resultats}:~\nameref{c:Resultats}.
La memòria es tanca amb el capítol~\ref{c:conclusions}:~\nameref{c:conclusions} on exposem les conclusions i les línies de recerca futura que són fruit del nostre TR.
%%%%%%%% Això no toca aquí!!!
% Tot i el treball constant i l’esforç que hem invertit, l’experiència ha estat majoritàriament positiva. Gràcies a aquest procés hem pogut enriquir-nos amb coneixements que no s’adquireixen a classe i guanyarem un avantatge en iniciar els estudis universitaris. A més, aquest treball ens ajudarà a millorar la redacció i el nivell de la llengua, que és un dels nostres punts febles.


%%%%%%%%%%%%%%%%%%%%%%%%%%%%%%%%%%%%%%%%%%%%
% Una cosa, crec que el capítol de xarxes neuronals hauria d'estar a dins de resultats i una altra part als annexos.

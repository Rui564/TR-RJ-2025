\chapter{Introducció}
\label{c:intro}
En el nostre dia a dia consumim continguts digitals de manera constant: visualitzem vídeos curts, fem fotografies, realitzem revisions corporals, juguem a videojocs, naveguem per Internet, entre moltes altres coses. Darrere de totes aquestes activitats hi trobem la presència de la intel·ligència artificial (IA), que té un paper rellevant en àmbits tan diversos com els l de continguts audiovisuals, el diagnòstic mèdic, la gestió i la distribució de la informació o l’optimització de la qualitat d’imatges.

Aquest protagonisme fa que, actualment, la IA sigui cada vegada més demandada en els entorns industrials i professionals. Tanmateix, la falta de formació específica i la complexitat pròpia del camp dificulten el seu accès.

Amb aquesta situació com a punt de partida, la nostra intenció és introduir-nos en el camp de la intel·ligència artificial, aprofitant l’oportunitat que ens ofereix el treball de recerca i amb la voluntat de treure’n el màxim profit. La nostra motivació principal és adquirir una formació que ens permeti preparar-nos per als reptes que ens depara el futur i obtenir una formació que ens ajudarà en el nostre futur acadèmic i professional.

%%%%%%%%%%%%%%%%%%%%%%%%%%%%%%%%%%%%%%%%%%%%%%%%%%%%%%%%%%%%%%%%%%%%%%%
%%%%%%%%%%%%%%%%%%%%%%%%%%%%%%%%%%%%%%%%%%%%%%%%%%%%%%%%%%%%%%%%%%%%%%%

\bigskip

\textbf{Cal afegir:\\ Cal introduir el treball d'una manera més específica\\\\
En el nostre treball ens centrarem en l'estudi i el desenvolupament de les xarxes neuronals que són peça clau en el ...}

%%%%%%%%%%%%%%%%%%%%%%%%%%%%%%%%%%%%%%%%%%%%%%%%%%%%%%%%%%%%%%%%%%%%%%%
%%%%%%%%%%%%%%%%%%%%%%%%%%%%%%%%%%%%%%%%%%%%%%%%%%%%%%%%%%%%%%%%%%%%%%%


\section{Motivació}
Ara que ja hem presentat la temàtica del nostre Treball de Recerca (TR), explicarem quines són les motivacions que ens empenyen a dur-lo a terme:

\begin{itemize}
 \item La principal raó que ens va portar a aquest tema és la nostra afició per la informàtica. En un futur ens agradaria aprofundir-hi i continuar treballant-hi en un grau, màster o doctorat. Per això, aprofitem l’oportunitat del TR per començar a preparar-nos.

 \item Una altra raó que ens va animar a triar aquest tema són els recursos que tenim a la nostra disposició. D'una banda, el nostre tutor disposa de contactes amb professors de la facultat de Matemàtiques i de la facultat d'Informàtica de la Universitat de Barcelona i de la Universitat Rovira i Virgili, a més, un amic nostre ens va oferir recursos externs per elaborar la part pràctica de la xarxa neuronal amb el full de càlcul.
%
%  \item El nostre tutor ens va proposar diversos temes, i finalment vam escollir aquest per l’interès que ens va despertar.
%  A mès, desprès d'haber escollit aquest tema, el nostre tutor es va posar en contacte amb profesors d'universitats per proposar-nos els tipus de treballs pràctis que podriem escollir, gràcies a això teniem molts videos i contingut per veure.

 \item Conjuntament amb el tutor vam decidir que el TR havia de ser una oportunitat per a apendre a fer servir recursos que ens fossin útils en els nostres estudis universitaris: un entorn de treball col$\cdot$laboratiu, dominar llenguatges de programació, un editor de text adient per a textos científics.

 \item La voluntat de construir una eina amb una mirada cap al futur, aplicant els coneixements adquirits i amb la determinació de fer un treball rigorós i ambiciosament plantejat.
\end{itemize}

\section{Estructura de la memòria}

Després d'aquesta breu introducció, al capítol~\ref{c:objectius}:~\nameref{c:objectius}. es presenten els objectius del nostre TR. A continació, al capítol~\ref{c:recerca prèvia}:~\nameref{c:recerca prèvia}, es presenten els coneixements previs necessaris per a comprendre el funcionament de les xarxes neuronals.
Després, vam iniciar una nova recerca per decidir quines eines utilitzar, amb l’objectiu de guanyar temps i millorar la qualitat del treball. Tot i que no sigui d’una complexitat excessiva, és un treball que requereix esforç i dedicació. Aquesta part es mostra en el capítol~\ref{c:Metodologia}:~\nameref{c:Metodologia}. Un cop hem presentat com treballarem, toca presentar els resultats obtinguts al ~\ref{c:Resultats}:~\nameref{c:Resultats}.
La memòria es tanca amb el capítol~\ref{c:conclusions}:~\nameref{c:conclusions} on exposem les conclusions i les línies de recerca futura que són fruit del nostre TR.
%%%%%%%% Això no toca aquí!!!
% Tot i el treball constant i l’esforç que hem invertit, l’experiència ha estat majoritàriament positiva. Gràcies a aquest procés hem pogut enriquir-nos amb coneixements que no s’adquireixen a classe i guanyarem un avantatge en iniciar els estudis universitaris. A més, aquest treball ens ajudarà a millorar la redacció i el nivell de la llengua, que és un dels nostres punts febles.


%%%%%%%%%%%%%%%%%%%%%%%%%%%%%%%%%%%%%%%%%%%%
% Una cosa, crec que el capítol de xarxes neuronals hauria d'estar a dins de resultats i una altra part als annexos.
